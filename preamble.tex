\usepackage{a4wide}
\usepackage[amssymb]{SIunits}
\usepackage{units}
\usepackage{pslatex}
\usepackage{amssymb}
\usepackage{amsbsy}
\usepackage[fleqn]{amsmath}
\usepackage{bbm}
\usepackage{mathtools}
\usepackage{booktabs}
\usepackage{url}
\usepackage[many]{tcolorbox} %Achtung! Reihenfolge beachten
\usepackage{xcolor}
\usepackage[utf8]{inputenc}
\usepackage[ngerman]{babel} % needs: sudo apt-get install texlive-lang-german
\usepackage[figurename=Abb.]{caption}
\usepackage{calrsfs}
\usepackage[titletoc]{appendix}
\usepackage{thumbpdf}
\usepackage[colorlinks,
pdfpagelabels,
pdfstartview = FitH,
bookmarksopen = true,
bookmarksnumbered = true,
linkcolor = black,
plainpages = false,
hypertexnames = false,
citecolor = black]{hyperref}
% Define the headers
\usepackage{fancyhdr}
\pagestyle{fancy}
\fancyhf{}
\fancyhead[EL]{\leftmark}
\fancyhead[OR]{\rightmark}
\fancyfoot[OR]{\thepage}
\fancyfoot[EL]{\thepage}
\renewcommand{\headrulewidth}{0pt}
% This is to have text flow around figures
\usepackage{wrapfig}
% Use tikz and pgfplot thru gnuplot
\usepackage[miktex]{gnuplottex}
\usepackage{gnuplot-lua-tikz}
\usepackage{tikz}
\usepackage{pgfplots}
%\pgfplotsset{compat=1.14}
\usetikzlibrary{decorations.pathreplacing,decorations.markings} %,snakes}
\usetikzlibrary{shapes,arrows,chains,matrix,positioning,scopes}
\usetikzlibrary{positioning}
\usetikzlibrary{calc}
\usetikzlibrary{arrows.meta}
\tikzset{darkstyle/.style={circle,draw,fill=gray!40}}
%colours
\definecolor{darkgreen}{rgb}{0,0.6,0}
\newcommand{\dgreen}{\color{darkgreen}}
\newcommand{\red}{\color{red}}
\newcommand{\blue}{\color{blue}}
\newcommand{\cyan}{\color{cyan}}
\newcommand{\green}{\color{green}}
\newcommand{\mbs}[1]{\ensuremath{\boldsymbol{\mathbf{#1}}}}
% New command and renewcommand settings
\newcommand{\mbf}{\mathbf}
\newcommand{\bs}{\boldsymbol}
%\renewcommand{\thepage}{{\thepart}.\arabic{page}}
%\addto\captionsgerman{\renewcommand{\figurename}{Abb.}}
\newtheorem{theorem}{Satz}
% New environment settings
%
\definecolor{myblue}{RGB}{0,163,243}
\tcbset{mystyle/.style={ 
%  breakable, 
  enhanced, 
  outer arc=0pt, 
  arc=0pt, 
  colframe=gray, 
  colback=white, 
  boxed title style={ 
    colback=gray,
    coltitle=black,
    colbacktitle=gray,
  }, 
  outer arc=5pt,
  title=Boxhead~\thetcbcounter, 
  fonttitle=\sffamily 
  } 
}
%
\newtcolorbox[auto counter]{satz}[1]{mystyle, title=Satz~\thetcbcounter: #1}
\newtcolorbox[auto counter]{exercise}[1]{mystyle, title=Aufgabe~\thetcbcounter: #1}
\newtcolorbox[auto counter]{example}[1]{breakable,mystyle, title=Beispiel~\thetcbcounter: #1}
\newtcolorbox{note}[1]{ enhanced,attach boxed title to top left={yshift=-3mm,yshifttext=-1mm},
  title={N.B.: #1},fonttitle=\bfseries,
  boxed title style={size=small}}
%
\newboolean{showComments} %Textblöcke anzeigen
\newcommand\Comment[1]{\ifthenelse{\boolean{showComments}}{\leavevmode\newline{\blue #1}\newline}{}}
%
\newboolean{showErratum} %Textblöcke anzeigen
\newcommand\Erratum[1]{\ifthenelse{\boolean{showErratum}}{\leavevmode\newline{\red #1}\newline}{}} 
%
%\title{\huge Lecture Notes\\
%\large für Studierende der Studiengänge Embedded Systems %Engineering, Mirkosystemtechnik und Sustainable Systems %Engineering}
%\author{Andreas Greiner und Lars Pastewka \\Professur für %Simulation\\Institut für Mikrosystemtechnik\\Universität %Freiburg}
\parindent0em
\parskip1ex
\pagestyle{headings}
\setcounter{secnumdepth}{3}
% The following is better placed in the main document
%\setboolean{showComments}{true}
%\setboolean{showErratum}{true}
\parindent0em
\parskip1ex
\pagestyle{headings}