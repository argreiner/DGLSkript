\chapter{Integraltransformationen}
Eine Integraltransformation der Form
\begin{equation}
  \mathcal{F}(y)=\int\limits_a^bK(y,x)f(x)dx
  \label{eq:IntTrafo}
\end{equation}
ist gegeben als ein linearer Operator $\int\limits_a^bK(y,x)\,.\,dx$, der auf
eine Funktion $f(x)$ wirkt.
\section{Die Fourierreihen}
Erinnern wir uns zunächst an die Fourierreihe: Jede stückweise stetige und im
Intervall $I=[-L/2,L/2]$ quadratintegrable Funktion $f(x)$ ist durch eine
Fourierreihe darstellbar:
\begin{equation}\label{eq:fourierreihe}
f(x)=\sum_{n=-\infty}^{n=\infty}f_n e^{ik_n x}\mbox{, mit }k_n=\frac{2\pi n}{L}
\end{equation}

Die Funktionen $e^{ik_nx}$ sind auf dem Intervall $I=[-L/2,L/2]$ orthogonal zueinander, d.h.
es gilt
\[ \int_{-\frac{L}{2}}^{\frac{L}{2}}e^{-ik_nx}e^{ik_mx}dx=L\delta_{mn}\]
Indem wir in
\begin{equation}\label{eq:fm}
f_m=\frac{1}{L}\int_{-\frac{L}{2}}^{\frac{L}{2}}e^{-ik_mx}f(x)dx
\end{equation}
(\ref{eq:fourierreihe}) einsetzen erhalten wir 
\begin{equation}\label{eq:fm}
f_m=\frac{1}{L}\int_{-\frac{L}{2}}^{\frac{L}{2}}e^{-ik_mx}f(x)dx=\frac{1}{L}
\sum_{n=-\infty}^{n=\infty}f_n\int_{-\frac{L}{2}}^{\frac{L}{2}}e^{-ik_mx}e^{ik_nx}dx
=L\delta_{mn}f_n
\end{equation}
Es gehören also die beiden Darstellungen im Raum der Wellenzahlen und dem Ortsraum zusammen
\begin{align}
  f(x)&=\sum_{n=-\infty}^{n=\infty}f_n e^{ik_n x}\\
f_m&=\frac{1}{L}\int_{-\frac{L}{2}}^{\frac{L}{2}}e^{-ik_mx}f(x)dx
\end{align}
\begin{note}{Quadratintegrabel}
Als quadratintegrabel wird eine reelle oder komplexwertige Funktion $f(x)$ auf
einem Intervall $I=[a,b]$ dann bezeichnet, wenn das Integral des Quadrats des
Absolutbetrags der Funktion über $I$ existiert und konvergiert, also
\[\int_a^b |f(x)|^2dx<\infty\]
\end{note}
%
\section{Die Fouriertransformation}
Mit (\ref{eq:fm}) in (\ref{eq:fourierreihe}) erhalten wir
\begin{equation}
f(x)=\sum_{n=-\infty}^{n=\infty}f_n e^{ik_n x}=\frac{1}{L}\sum_{n=-\infty}^{n=\infty}
e^{ik_nx}\int_{-\frac{L}{2}}^{\frac{L}{2}}f(x') e^{-ik_nx'}dx'
\end{equation}
Wir schreiben
\[
\Delta k_n=k_n-k_{n-1}=\frac{2\pi}{L}
\]
und erhalten damit
\[
f(x)=\frac{1}{2\pi}\sum_{n=-\infty}^{n=\infty}
\underbrace{\int_{-\frac{L}{2}}^{\frac{L}{2}}f(x') e^{-ik_nx'}dx'}_{\mbox{im Limes} (L\rightarrow\infty)\rightarrow F(k)}
e^{ik_nx}\Delta k_n
\]
Wir machen den Grenzübergang $L\rightarrow\infty$ und erhalten damit
\[
f(x)=\frac{1}{2\pi}\int\limits_{-\infty}^{\infty}F(k) e^{ikx}dk
\]

Damit erhalten wir die Fouriertransformation und ihre Inverse als
\begin{eqnarray}\label{eq:fouriertrafo}
{\cal F}\{f(x)\}=F(k)=\int_{-\infty}^{\infty}e^{-ikx}f(x)dx\label{eq:fourierhin}\\
{\cal F}^{-1}\{F(k)\}=f(x)=\frac{1}{2\pi}\int_{-\infty}^{\infty}e^{ikx}F(k)dk\label{eq:fourierrueck}
\end{eqnarray}
\begin{note}{Verschiedene Formen der Fouriertransformation}
	\begin{enumerate}
		\item Frequenz $\nu$ (Hertz) und unitär
		\begin{eqnarray*}
			\hat{f}_1(\bs{\nu})\ &\stackrel{\mathrm{def}}{=}&\
			\int_{\mathbb{R}^n} f(\mbf{x}) e^{-2 \pi i \bs{\nu}\cdot\mbf{x}}\, d^nx =
			\hat{f}_2(2 \pi \bs{\nu})=(2 \pi)^{n/2}\hat{f}_3(2 \pi \bs{\nu})\\
			f(\mbf{x}) &=& \int_{\mathbb{R}^n} \hat{f}_1(\bs{\nu}) e^{2 \pi i
			\bs{\nu}\cdot\mbf{x}}\, d^n\nu 
		\end{eqnarray*}
		\item Kreisfrequenz $\omega$ (rad/s) und nicht unitär
		\begin{eqnarray*}
			\hat{f}_2(\bs{\omega}) \
			&\stackrel{\mathrm{def}}{=}&\int_{\mathbb{R}^n} f(\mbf{x})
			e^{-i\bs{\omega}\cdot \mbf{x}} \, d^nx \ = \hat{f}_1 \left (
			\frac{\bs{\omega}}{2 \pi} \right ) = (2 \pi)^{n/2}\
			\hat{f}_3(\bs{\omega})\\ 
			f(\mbf{x}) &=& \frac{1}{(2 \pi)^n} \int_{\mathbb{R}^n}
			\hat{f}_2(\bs{\omega}) e^{i \bs{\omega}\cdot \mbf{x}} \, d^n\omega 
		\end{eqnarray*}
		\item unitär
		\begin{eqnarray*}
			\hat{f}_3(\bs{\omega}) \ &\stackrel{\mathrm{def}}{=}&\
			\frac{1}{(2 \pi)^{n/2}} \int_{\mathbb{R}^n} f(\mbf{x}) \
			e^{-i \bs{\omega}\cdot \mbf{x}}\, d^nx = 
			\frac{\hat{f}_1(\bs{\omega}/2\pi)}{(2 \pi)^{n/2}}
			 =
			\frac{\hat{f}_2(\bs{\omega})}{(2 \pi)^{n/2}} \\ 
			f(\mbf{x}) &=& \frac{1}{(2 \pi)^{n/2}} \int_{\mathbb{R}^n}
			\hat{f}_3(\bs{\omega})e^{i \bs{\omega}\cdot \mbf{x}}\, d^n\omega
		\end{eqnarray*}
	\end{enumerate}
\end{note}
Wir verwenden im folgenden die 2. Form der Fouriertransformation.

An Unstetigkeiten muss die Funktion die Bedingungen 
\[
  \lim_{\varepsilon\rightarrow 0}\frac{f(t+\varepsilon)+f(t-\varepsilon)}{2}=
  \frac{1}{2\pi}\int_{-\infty}^{\infty}e^{ik_nx}F(k)dk
\]  
erfüllen.
\begin{example}{Rechteck}
Gegeben die Funktion
  \[  
  f(t)=\left\{
    	\begin{array}{ccl}
    		1&\mbox{f\"ur}&0\le|t|<1\\ 
    		\frac{1}{2}&\mbox{f\"ur}&|t|=1\\
    		0&\mbox{f\"ur}&|t|>0 
         \end{array}
	\right.
\]
Dann ist die Fouriertransformierte
\[
  F(\omega)=\int_{-\infty}^{\infty}e^{-i\omega t}f(t)dt=
  2\int_{0}^{1}f(t)\cos(\omega t)dt=\frac{\sin(\omega)}{\omega}
\]
\end{example}
\section{Die Laplacetransformation}
Wir betrachten die Fouriertransformation der auf einem Halbraum definierten
Funktion $f(t)\mbox{, }t\in[0,\infty)$. Voraussetzung ist, dass $f(t)\cdot
e^{-tx}$, für $x>0$ im Intervall $[0,\infty)$ quadratintegrabel ist. Also ist
deren Fouriertransformation
\begin{equation}\label{eq:prelaplacetrafo}
{\cal F}\{e^{-xt}f(t)\}=F(y)=\int_0^\infty e^{-iyt}e^{-xt}f(t)dt
\end{equation}
Setzen wir $s=x+iy$ und beschränken wir uns auf $\Re(s)>0$ dann können wir für die 
Funktion $f(t)$ folgende Integraltransformatin definieren 
\begin{equation}\label{eq:Laplacetrafo}
  {\cal L}\{f(t)\}=F(s)=\int_0^\infty f(t) e^{-st}dt
\end{equation}
Aufgrund des positiven Realteils von $s$ konvergiert das Integral für $t>0$
und wir nennen $F(s)$ die {\bf Laplacetransformierte} von $f(t)$.

Die Umkehrung der Laplacetransformation erhalten wir aus der komplexen
Inversionsformel
\begin{equation}\label{eq:Laplaceinverse}
  {\cal L}^{-1}\{F(s)\}=f(t)=\frac{1}{2\pi i}\lim_{\eta\rightarrow\infty}
\int_{\gamma-i\eta}^{\gamma+i\eta} F(s) e^{st}ds
\end{equation}
%
\begin{figure}[ht]
\centering
\begin{tikzpicture}
\draw[thick,-{Latex[scale=1.5]}] (-5.5,0) -- (5.5,0);
\node at (5.8,0) {\large\bf x};
\draw[thick,-{Latex[scale=1.5]}] (0,-5.5) -- (0,5.5);
\node at (0,5.8) {\large\bf y};
\draw[thick,-{Latex[scale=2.0]}] (3,-4) -- (3,2);
\draw[thick] (3,2) -- (3,4);
\draw[fill=black] (3,-4) circle (0.1);
\node (A) at (4.5,-4) {\large $\boldsymbol{\gamma}-i\boldsymbol{\eta}$};
\node (A) at (3,-4.6) {\large\bf A};
\draw[fill=black] (3,4) circle (0.1);
\node (B) at (4.5,4) {\large $\boldsymbol{\gamma}+i\boldsymbol{\eta}$};
\node (B) at (3,4.5) {\large\bf B};
\node (G) at (-5,1.5) {\large $\boldsymbol{\Gamma}$};
\draw[thick] (0,0) -- (3,4);
\node (R) at (1.3,2.5) {\large\bf R};
 \draw[thick,-{Latex[scale=2.0]}] (3, 4) arc(53.13:135:5);
 \draw[thick,-{Latex[scale=2.0]}]  (-3.536,3.536) arc(135:225:5);
 \draw[thick]  (-3.536,-3.536) arc(225:306.87:5);
\end{tikzpicture}
%\includegraphics[width=0.5\textwidth]{Figures/Bromwich.pdf}
\caption{\label{fig:Bromwich}Die Bromwich-Kurve}
\end{figure}
wobei wir $f(t)=0$ für $t<0$ verlangen und $\gamma$ so wählen, dass alle
Singularitäten links davon liegen. Wir integrieren also entlang der Geraden
$s=\gamma+iy$ von $y=-\eta$ bis $y=+\eta$ und machen den Gernzübergang für
$\eta\rightarrow\infty$. Die Berechnung des Integrals lässt sich wesentlich
einfacher gestalten mit der sogenannten Bromwich-Kurve. Die Motivation hierfür
ist, dass man den Residuensatz ausnutzen kann. Denn wenn wir die in Abbildung
\ref{fig:Bromwich} anschauen, dann sehen wir dort den Integrationsweg, der in
(\ref{eq:Laplaceinverse}) benutzt wurde. Diesen ergänzen wir mit dem
Integrationsweg $\Gamma$ und erhalten eine geschlossene positiv orientierte
Kurve $\cal C$ bestehend aus der Strecke von A nach B und dem Kreisausschnitt
$\Gamma$, über welche wir die Funktion $F(s)$ integrieren können
\begin{equation}\label{eq:Kreisintegral}
c(t)=\frac{1}{2\pi i}\oint_{\cal C}e^{st}F(s)ds
\end{equation}
Dieses Integral kann mit Hilfe des Residuensatzes (siehe Anhang \ref{sec:Residuensatz}) ausgewertet werden.  
%(siehe Anhang \ref{chap:Residuensatz}).
Nun haben wir aber einen Fehler gemacht, den wir korrigiren müssen, das Integral
über den Weg $\Gamma$ muss wieder abgezogen werden, damit wir die inverse
Laplacatransformation, wie in (\ref{eq:Laplaceinverse}) rekonstruieren
\begin{equation}\label{eq:Laplaceinverse2}
f(t)=\frac{1}{2\pi i}\lim_{R\rightarrow\infty}\left[
\underbrace{\oint_{\cal C}e^{st}F(s)ds}_{\text{Residuensatz}}
-\underbrace{\int_{\Gamma} F(s) e^{st}ds}_{\rightarrow 0}\right]
\end{equation}
Das zweite Integral geht aber unter bestimmten Bedingungen gegen Null. 
Dies ist immer der Fall, wenn auf $\Gamma$
\[|F(s)|<\frac{M}{R^k}\]
gilt, wobei $M>0$ und $k>0$ ist.

\begin{example}{Anwendung}
Wir wenden die Laplacetransformation auf das Anfangswertproblem
\begin{equation}
  \frac{d^2y(t)}{dt^2}+ \frac{dy(t)}{dt}+2y(t)=g(t), 
  \label{eq:LTexample1}
\end{equation}
\[\mbox{ mit } y(0)=y_0\mbox{ und } \left.\frac{dy(t)}{dt}\right|_{t=0}=\dot{y}_0\mbox{ an.}\]
Durch partielle Integration können wir zeigen, dass gilt
\[\int_{t_0}^\infty\frac{dy(t)}{dt}e^{-st}dt=
\left.y(t)e^{-st}\right|_0^\infty+s\int_{t_0}^\infty y(t)e^{-st}dt=-y_0+sY(s)\]
Durch zweimalige partielle Integration zeigen wir, dass gilt
\[\int_{t_0}^\infty\frac{d^2y(t)}{dt^2}e^{-st}dt=-\dot{y}_0-sy_0+s^2Y(s)\]
Des weiteren gilt ${\cal L}(g(t))=G(s)$. Damit erhalten wir bei Anwendung der
Laplacetransformation auf sämtliche Terme von (\ref{eq:LTexample1})
\[
\underbrace{s^2Y(s)+sY(s)+2Y(s)}_{\text{Differentialoperator}}
\underbrace{-y_0-\dot{y}_0-sy_0}_{\text{Anfangsedingungen}}=
\underbrace{G(s)}_{\text{Inhomogenität}}
\]  
oder
\begin{equation}
  Y(s)=\underbrace{\frac{G(s)}{s^2+s+2}}_{\mbox{inhom. Lösung}}
  +\underbrace{\frac{(s+1)y_0+\dot{y}_0}{s^2+s+2}}_{\mbox{hom. Lösung+Anfangsb.}}
\end{equation}
\end{example}
\newpage