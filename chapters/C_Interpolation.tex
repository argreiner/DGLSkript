\chapter{Interpolation und Kurvenanpassung}

\section{Least square fit}
Suppose you are given a set of data $z_i\in\{Z\}$, e.g. the $i=1,\dots,29$ indicating the channels of an EEG measurement, measured at points $x_i$ and $y_i$. We want to find a function $f(x,y;\mathbf{p})$ which depends on a vector $\mathbf{p})$ of paramters $p_k$. 
\subsection{Polynomial basis}
Assume that $f(x=0,y=0)=0$. Thus we propose a polynomial function $f(x,y;\mathbf{p})=p_0\cdot x+p_1\cdot y+p_2\cdot x^2+p_3\cdot x\cdot y+p_4\cdot y^2$ the data set of measurements $\{x_i,y_i,z_i\}$ we want to fit to. We choose a least square method, i.e.
\[
\sum_i \left( f(x_i,y_i;\mathbf{p})-z_i\right)^2 = \text{Min}.
\]
Which in turn is equivalent to require
\[
\frac{\partial}{\partial p_k}\sum_i \left( f(x_i,y_i;\mathbf{p})-z_i\right)^2=\sum_i \left( f(x_i,y_i;\mathbf{p})-z_i\right)\frac{\partial}{\partial p_k}f(x_i,y_i;\mathbf{p})=0
\]
this yields a linear system of equation for the $p_k$ to read
\begin{align*}
p_0\sum_i x_i^2+p_1\sum_i x_i\cdot y_i+p_2\sum_i x_i^3+p_3\sum_i x_i^2\cdot y_i+p_4\sum_i x_i\cdot y_i^2&=\sum_iz_i\cdot x_i\\
p_0\sum_i x_i\cdot y_i+p_1\sum_i y_i^2+p_2\sum_i x_i^2\cdot y_i+p_3\sum_i x_i\cdot y_i^2+p_4\sum_i y_i^3&=\sum_iz_i\cdot y_i\\
p_0\sum_i x_i^3+p_1\sum_i x_i^2\cdot y_i+p_2\sum_i x_i^4+p_3\sum_i x_i^3\cdot y_i+p_4\sum_i x_i^2\cdot y_i^2&=\sum_iz_i\cdot x_i^2\\
p_0\sum_i x_i^2\cdot y_i+p_1\sum_i x_i\cdot y_i^2+p_2\sum_i x_i^3\cdot y_i+p_3\sum_i x_i^2\cdot y_i^2+p_4\sum_i x_i\cdot y_i^3&=\sum_iz_i\cdot x_i\cdot y_i\\
p_0\sum_i x_i\cdot y_i^2+p_1\sum_i y_i^3+p_2\sum_i x_i^2\cdot y_i^2+p_3\sum_i x_i\cdot y_i^3+p_4\sum_i x_i\cdot y_i^4&=\sum_iz_i\cdot y_i^2
\end{align*}
Let us write this as a 
\[
\mathbf{A}\cdot\mathbf{p}=\mathbf{b}
\]
Note that $\mathbf{A}$ is symmetric and the matrix entries read
\[ 
\mathbf{A}=
\begin{pmatrix}
\sum_i x_i^2&\sum_i x_i\cdot y_i&\sum_i x_i^3&\sum_i x_i^2\cdot y_i&\sum_i x_i\cdot y_i^2\\
\sum_i x_i\cdot y_i&\sum_i y_i^2&\sum_i x_i^2\cdot y_i&\sum_i x_i\cdot y_i^2&\sum_i y_i^3\\
\sum_i x_i^3&\sum_i x_i^2\cdot y_i&\sum_i x_i^4&\sum_i x_i^3\cdot y_i&\sum_i x_i^2\cdot y_i^2\\
\sum_i x_i^2\cdot y_i&\sum_i x_i\cdot y_i^2&\sum_i x_i^3\cdot y_i&\sum_i x_i^2\cdot y_i^2&\sum_i x_i\cdot y_i^3\\
\sum_i x_i\cdot y_i^2&\sum_i y_i^3&\sum_i x_i^2\cdot y_i^2&\sum_i x_i\cdot y_i^3&\sum_i x_i\cdot y_i^4
\end{pmatrix},
\]
while we habe for the r.h.s. of the linear system
\[ 
\mathbf{b}=
\begin{pmatrix}
\sum_iz_i\cdot x_i\\
\sum_iz_i\cdot y_i\\
\sum_iz_i\cdot x_i^2\\
\sum_iz_i\cdot x_i\cdot y_i\\
\sum_iz_i\cdot y_i^2
\end{pmatrix},
\]
$\mathbf{A}$ and $\mathbf{b}$ can be easily calculated and thus we have $\mathbf{p}=\mathbf{A}^{-1}\cdot\mathbf{b}$. The only task left is to find an analytical expression for the inverse of $\mathbf{A}$, which components read
\[
\mathbf{A}^{-1}_{ij}=\frac{\alpha_{ij}}{|\mathbf{A}|},
\]
where the $\alpha_{ij}$ are the elements of the adjoint matrix $\mathbf{A}_{ad}$. 
\begin{note}{The adjoint matrix elements}
\[
\alpha_{ij}=(-1)^{i+j}
\left|
\begin{matrix}
a_{11}&\dots &a_{1,j-1}& &a_{1,j+1}&\dots &a_{1n}\\
a_{11}&\dots &a_{1,j-1}& &a_{1,j+1}&\dots &a_{1n}\\
&&&\vdots &&&\\
a_{11}&\dots &a_{1,j-1}& &a_{1,j+1}&\dots &a_{1n}\\
a_{11}&\dots &a_{1,j-1}& &a_{1,j+1}&\dots &a_{1n}\\
\end{matrix}
\right|\text{, where }a_{ij}=(\mathbf{A})_{ij}
\]
\end{note}
\subsection{Spherical harmonics function as a basis}
Now let $z_i\in\{Z\}$ be the $i=1,\dots,29$ indicating the channels of an EEG measurement at positions $\theta_i(t),\varphi_i(t)$, i.e. 29 time series'.
We propose a set of spherical harmonics functions $f(x,y;\mathbf{p})=Y_{nl}(\theta,\varphi)$ the data set of measurements $\{\theta_i,\varphi_i,z_i\}$ we want to fit to. The least square method states that
\[
\sum_i \left(\sum\limits_{nl}c_{nl}Y_{nl}(\theta_i,\varphi_i)-z_i\right)^2 = \text{Min}.
\]
Thus all the derivatives w.r.t. $c_{mk}$ must vanish, to read
\begin{equation}\label{eq:lstsqsph}
    \sum\limits_i2\left(\sum\limits_{nl}c_{nl}Y_{nl}(\theta_i,\varphi_i)-z_i\right)\cdot Y_{mk}(\theta_i,\varphi_i)=0
\end{equation}
For the spherical harmonics we have $l=-n,\dots,n$ for every $n$.
\begin{example}{Spherical harmonics least square fit}
Let us limit the number of spherical harmonics in (\ref{eq:lstsqsph}) to $n=0,1$. Thus we are left with 4 parameters $c_0=c_0,c_{1,-1}=c_1,c_{1,0}=c_2,c_{1,1}=c_4$. Thus (\ref{eq:lstsqsph}) reads
\[
\sum\limits_{i=0}^{29}
\left(\sum\limits_{j=0}c_{j}Y_{j}(\theta_i,\varphi_i)
-z_i\right)\cdot Y_{k}(\theta_i,\varphi_i)=0
\]
We get four equations - remember $k=0,1,2,3$
\begin{align*}
&c_{0}\sum\limits_{i}Y_{0}(\theta_i,\varphi_i)\cdot Y_{k}(\theta_i,\varphi_i)+
c_{1}\sum\limits_{i}Y_{1}(\theta_i,\varphi_i)\cdot Y_{k}(\theta_i,\varphi_i)+\\
&c_{2}\sum\limits_{i}Y_{2}(\theta_i,\varphi_i)\cdot Y_{k}(\theta_i,\varphi_i)+
c_{3}\sum\limits_{i}Y_{3}(\theta_i,\varphi_i)\cdot Y_{k}(\theta_i,\varphi_i)
=\sum\limits_{i}z_iY_{k}(\theta_i,\varphi_i)
\end{align*}
The $a_{kj}=\sum\limits_{i}Y_{j}(\theta_i,\varphi_i)\cdot Y_{k}(\theta_i,\varphi_i)$ and the $b_k=\sum\limits_{i}z_iY_{k}(\theta_i,\varphi_i)$ are just real numbers. Thus we end up with the linear system
\begin{align*}
    a_{00}c_0+a_{01}c_1+a_{02}c_2+a_{03}c_3=&b_0\\
    a_{10}c_0+a_{11}c_1+a_{12}c_2+a_{13}c_3=&b_1\\
    a_{20}c_0+a_{21}c_1+a_{22}c_2+a_{23}c_3=&b_2\\
    a_{30}c_0+a_{31}c_1+a_{32}c_2+a_{33}c_3=&b_3\\
\end{align*}
to determine the $c_i$.
\end{example}