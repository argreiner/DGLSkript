\chapter{Systeme linearer gewöhnlicher Differentialgleichungen}
Wir benutzen die Operatorschreibweise, um ein allgemeines System von
Differentialgleichungen beliebiger Ordnung anzugeben. Wir betrachten somit
Systeme der Art
\begin{equation}
  P_{i_1}(D)[y_1(t)]+P_{i_2}(D)[y_2(t)]+\dots+P_{i_n}(D)[y_n(t)]=f_i(t)\hfill
  (i=1,2,\dots,m)
  \label{eq:SystemGeneral}
\end{equation}
Um (\ref{eq:SystemCs}) zu lösen gibt es zwei Strategien: Erstens könnten wir
aus dem vorliegenden System eine Differentialgleichung höherer Ordnung machen
oder zweitens könnten wir versuchen auf ein System von gekoppelten
Differentialgleichungen erster Ordnung überzugehen.

Anhand zweier Beispiele wollen wir die Vor- und Nachteile beider Methoden
untersuchen und aus den Nachteilen gegebenenfalls Rückschlüsse auf die Modelle
ziehen.
\begin{example}{Rückführung auf DGL höherer Ordnung}
  Ähnlich dem Gauß'schen Eliminationsverfahren schreiben wir
  \begin{equation*}
    \begin{array}{rrl|l}
      D^2[y_1(t)]+&y_2(t)&=0&D^2\\
      y_1(t)+&D^2[y_2(t)]&=0&\cdot(-1)
    \end{array}
  \end{equation*}
  und erhalten die DGL $D^4[y_1(t)]-y_1(t)=0$.

Versuchen wir die Methode hingegen auf das System
\begin{align*}
  (D+1)[y_1(t)]+(D^2-1)[y_2(t)]&=0\\
  y_1(t)+(D-1)[y_2(t)]&=0
\end{align*}
anzuwenden, gelingt die Elimination von $y_1(t)$ oder $y_2(t)$ nicht, denn die
beiden Gleichungen sind linear abhängig (warum?).
\end{example}
Wie wir aus dem Gegenbeispiel gesehen haben, ist die Tatsache der linearen
Abhängigkeit auf eine nicht eindeutige Formulierung der Problemstellung
zurückzuführen. Denn entweder ist $y_1(t)$ oder $y_2(t)$ frei wählbar und die
Gleichung dann nach der jeweils anderen Funktion zu lösen. Diese Wahlfreiheit
bedeutet aber einen Mangel an Information über das Modellsystem. Für eine
eindeutige Lösung müssen wir uns Gedanken über weitere Bedingungen machen, die
wir dem System auferlegen, damit es eindeutig mit Hilfe der
Differentialgleichungen formuliert werden kann. Das beudeutet dann noch nicht,
dass es lösbar sein wird.
\begin{example}{Rückführung auf ein System linearer DGL 1. Ordnung}
  Aus dem System
  \begin{align*}
    \ddot{y}_1(t)+y_2(t)=0\\
    \ddot{y}_2(t)+y_1(t)=0
  \end{align*}
  erhalten wir mit $\dot{y}_1(t)=y_3(t)$ und $\dot{y}_2(t)=y_4(t)$ das System
  \begin{align*}
    \dot{y}_1(t)=y_3(t)\\
    \dot{y}_2(t)=y_4(t)\\
    \dot{y}_3(t)=-y_2(t)\\
    \dot{y}_4(t)=-y_1(t)
  \end{align*}
  Versuchen wir allerdings diese Methode auf das System
    \begin{align*}
      \ddot{y}_2(t)+\dot{y}_1(t)+y_1(t)-y_2(t)=0\\
      \dot{y}_2(t)+y_1(t)-y_2(t)=0
  \end{align*}
  anzuwenden, dann lässt sich dieses nicht in ein System der gewünschten Art
  überführen. In diesem Fall lässt sich die Funktion $y_1(t)$ frei wählen und
  dann das lineare System lösen. Da aber von vornherein $y_1(t)$ nicht
  eindeutig bestimmt ist, gilt das oben gesagte.
\end{example}
Wir werden uns im folgenden daher nur mit der Form
\[
  \dot{y}_i(t)=a_{i1}y_1(t)+a_{i2}y_2(t)+\dots+a_{in}y_n(t)\hfill (i=1,2,\dots,m)
\]
befassen.
\subsection{Homogene Systeme erster Ordnung}
Ein homogenes System von linearen Differentialgleichungen erster Ordnung hat
die Form
\begin{equation}
  \begin{pmatrix}
    \dot{y}_1(t)\\
    \dot{y}_2(t)\\
    \dots\\
    \dot{y}_n(t)
  \end{pmatrix}
  =
  \begin{pmatrix}
    a_{11}&a_{12}&\dots &a_{1n}\\
    a_{21}&a_{22}&\dots &a_{2n}\\
    \multicolumn{4}{c}\dotfill\\
    a_{n1}&a_{n2}&\dots &a_{nn}\\
  \end{pmatrix}
  \begin{pmatrix}
    y_1(t)\\
    y_2(t)\\
    \dots\\
    y_n(t)
  \end{pmatrix}
  \label{eq:System1stOrder}
\end{equation}
Das System (\ref{eq:System1stOrder}) lässt sich mit Hilfe der
Operatorschreibweise auch so darstellen
\begin{equation}
  \begin{pmatrix}
    a_{11}-D&a_{12}&\dots &a_{1n}\\
    a_{21}&a_{22}-D&\dots &a_{2n}\\
    \multicolumn{4}{c}\dotfill\\
    a_{n1}&a_{n2}&\dots &a_{nn}-D\\
  \end{pmatrix}
  \begin{pmatrix}
    y_1(t)\\
    y_2(t)\\
    \dots\\
    y_n(t)
  \end{pmatrix}
  =0
  \label{eq:System1stOperator}
\end{equation}
Oder wir schreiben es ganz einfach als
\begin{equation}
  \dot{\vec{y}}(t)=A\cdot\vec{y}(t)
  \label{eq:SystemVectorMatrixNotation}
\end{equation}
Es gibt genau eine Lösung $(y_1(t),y_2(t),\dots,y_n(t))$ von
(\ref{eq:System1stOrder}), so dass die zu einem beliebigen Zeitpunkt $t_0$
gewählten Bedingungen
\begin{equation}
  y_1(t_0)=b1,\:y_2(t_0)=b_2,\dots,y_n(t_0)=b_n
  \label{eq:ICsSystem}
\end{equation}
erfüllt sind. Wir werden den Beweis hier nicht antreten und verweisen
stattdessen auf die einschlägigen Lehrbücher.

Für die Lösung der Systeme (\ref{eq:System1stOrder}) benutzen wir den
Exponentialansatz $e^{\lambda t}$. Da wir $n$ Funktionen zu bestimmen haben,
schreiben wir diesen gleich in der Form
\begin{equation}
  y_i(t)=c_ie^{\lambda t}\hfill (i=1,2,\dots,n\mbox{ und }c_i\in\mathbb{R})
  \label{eq:SystemAnsatz}
\end{equation}
Wir setzen den Ansatz in (\ref{eq:System1stOrder}) ein und erhalten das
Gleichungssystem
\begin{align}
    (a_{11}-\lambda)c_1+a_{12}c_2+\dots+a_{1n}c_n&=0\nonumber\\
    a_{21}c_1+(a_{22}-\lambda)c_2+\dots+a_{2n}c_n&=0\nonumber\\
    \dots\dots\dots\dots\nonumber\\
    a_{21}c_1+a_{22}c_2+\dots+(a_{2n}-\lambda)c_n&=0
  \label{eq:System4cs}
\end{align}
Oder einfacher
\begin{equation}
  (A-\lambda\mathbbm{1})\cdot\vec{c}=0
  \label{eq:Vektorform}
\end{equation}
Aus der linearen Algebra wissen wir, dass (\ref{eq:Vektorform}) nur eine
Lösung hat, wenn gilt $|A-\lambda\mathbbm{1}|=0$ und die für die Lösungen
gewählten Werte der $\lambda$ sind die Eigenwerte der Koeffizientenmatrix $A$.

Um (\ref{eq:Vektorform}) zu lösen erinnern wir uns an Eigenwerte, Eigenvektoren
und Normalformen quadratischer Matrizen. Wir wissen es gibt drei Fälle:
\begin{enumerate}
  \item Alle Eigenwerte $\lambda_i$ sind voneinander verschieden.
  \item Es treten mehrfache Eigenwerte auf. Der Rang der Matrix
    $A-\lambda\mathbbm{1}$ fällt für jeden mehrfachen Eigenwert um soviel
    gegenüber $n$, wie die Vielfachheit des entsprechenden Eigenwertes beträgt.
  \item Es tritt mindestens ein mehrfacher Eigenwert auf für den der Rangabfall
    nicht die obere Regel erfüllt.
\end{enumerate}
Die ersten beiden Fälle erlauben uns die Matrix $A$ auf Diagonalform 
\begin{equation}
  \Lambda=
  \begin{pmatrix}
    \lambda_1&0&\dots&0\\
    0&\lambda_2&\dots&0\\
    \multicolumn{4}{c}\dotfill\\
    0&0&\dots&\lambda_n
  \end{pmatrix}
  \label{eq:Diagonalform}
\end{equation}
zu bringen. Was bedeutet das für unser Gleichungssystem? Wir schreiben
$\vec{y}^T$ als Transformierte eines Vektors $\vec{z}^T$ dergestalt
$\vec{y}^T=C\cdot\vec{z}^T$ und damit auch $\dot{\vec{y}}^T = C\cdot
\dot{\vec{z}}^T$. Damit erhalten wir 
\[ C\cdot \dot{\vec{z}}^T = A\cdot C\cdot\vec{z}^T \]
und wenn wir mit $C^{-1}$ multiplizieren ergibt sich
\begin{equation}
  \dot{\vec{z}}^T = C^{-1}A\cdot C\cdot\vec{z}^T
  \label{eq:Diagonalsystem}
\end{equation}
Der Vorteil erschließt sich uns sofort: wir lösen das einfache Diagonalsysteme
(\ref{eq:Diagonalsystem}) und transformieren dann zurück um $\vec{y}(t)$ zu
erhalten.
%

\textbf{Fall 1: }Wir bestimmen zunächst die Eigenvektoren $(c_{i1}, c_{i2},
\dots, c_{in})$, die zu den jeweiligen Eigenwerten $\lambda_i$ gehören. Diese
bilden die Spalten der Transformationsmatrix 
\begin{equation}
  C=\begin{pmatrix}
    c_{11}&c_{21}&\dots&c_{n1}\\
    c_{12}&c_{22}&\dots&c_{n2}\\
    &&\dots&\\
    c_{1n}&c_{2n}&\dots&c_{nn}
  \end{pmatrix}
  \label{eq:Tranformationsmatrix}
\end{equation}
mit deren Hilfe wir die Matrix $A$ in die Diagonalform $\Lambda$ übeführen können
\begin{equation}
  C^{-1}\cdot A\cdot C=\Lambda
%  \label{eq:Diagonalform}
\end{equation}
Mit (\ref{eq:Tranformationsmatrix}) können wir die allgemeine Lösung angeben
\begin{equation}
  \begin{pmatrix}y_1(t)\\y_2(t)\\\dots\\y_n(t)\end{pmatrix}=
  \begin{pmatrix}
    c_{11}d_1&c_{21}d_2&\dots&c_{n1}d_n\\
    c_{12}d_1&c_{22}d_2&\dots&c_{n2}d_n\\
    &&\dots&\\
    c_{1n}d_1&c_{2n}d_2&\dots&c_{nn}d_n\\
  \end{pmatrix}\cdot
  \begin{pmatrix}e^{\lambda_1t}\\e^{\lambda_2t}\\\dots\\e^{\lambda_nt}\end{pmatrix}
  \label{eq:LsgAllgFall1}
\end{equation}
Die beliebig wählbaren reellen Konstanten $d_i$ benutzen wir zur Erfüllung der
Anfangsbedingungen.

\textbf{Fall 2: }Es treten mehrfache Eigenwerte auf und der Rangabfall der
Koeffizientenmatrix des Gleichungssystems aus (\ref{eq:System4cs}) ist gleich
dem Vielfachen des Eigenwertes. Auch in diesem Fall lässt sich $A$ in die
Diagonalform bringen. Wir erhalten parametrische Lösungen für die
Eigenvektroren der mehrfachen Eigenwerte, die wir so wählen können, dass sie
linear unabhängig sind. Das weitere Vorgehen ist genauso, wie in Fall 1.

\textbf{Fall 3: }In diesem Fall können wir die Matrix $A$ nicht auf
Diagonalform jedoch auf die Jordansche Normalform bringen.
\begin{equation*}
  J=\begin{pmatrix}
    \lambda_1&\delta_1&0&\dots&0&0\\
    0&\lambda_2&\delta_2&\dots&0&0\\
    0&0&\lambda_3&\dots&0&0\\
    &&&\dots&&\\
    0&0&0&\dots&\lambda_{n-1}&\delta_{n-1}\\
    0&0&0&\dots&0&\lambda_n\\
  \end{pmatrix}
\end{equation*}
Die $\lambda_i$ sind die Eigenwerte der Matrix $A$ und die $\delta_i$ entweder
$0$ oder $1$. Es gibt eine Transformationsmatrix $C$, die $A$ auf die
Jordanform gemäß \[ C^{-1}\cdot A\cdot C=J \] abbildet. Für einen vielfachen
Eigenwert $\lambda_k$ mit der Multiplizität $k$ bilden wir aus den Lösungen der
Gleichungen $(A - \lambda_k \mathbbm{1})^2 \vec{c}^T = 0$, $(A - \lambda_k
\mathbbm{1})^3\vec{c}^T = 0$, \dots, $(A - \lambda_k \mathbbm{1})^k \vec{c}^T =
0$ linear unabhängige Vektoren, die wir zur Transformationsmatrix
zusammenbauen. Wir bringen das System (\ref{eq:System1stOrder}) auf die Jordanform und lösen dann die Gleichung
\begin{equation}
  \begin{pmatrix}\dot{z}_1\\\dot{z}_2\\\dot{z}_3\\\dots\\\dot{z}_{n-1}\\\dot{z}_{n}\end{pmatrix}=
  \begin{pmatrix}
    \lambda_1&\delta_1&0&\dots&0&0\\
    0&\lambda_2&\delta_2&\dots&0&0\\
    0&0&\lambda_3&\dots&0&0\\
    &&&\dots&&\\
    0&0&0&\dots&\lambda_{n-1}&\delta_{n-1}\\
    0&0&0&\dots&0&\lambda_n\\
  \end{pmatrix}
  \begin{pmatrix}z_1\\z_2\\z_3\\\dots\\z_{n-1}\\z_{n}\end{pmatrix}
  \label{eq:JordanNormalForm} 
\end{equation}
Wieder lösen wir das System (\ref{eq:JordanNormalForm}) nach $\vec{z}^T$ auf
und transformieren danach wieder zurück\newline $\vec{y}^T=C\cdot\vec{z}^T$. 
\subsection{Inhomogene Systeme erster Ordnung}
In diesem Abschnitt beschäftigen wir uns mit den inhomogenen Systemen von
Differentialgleichungen erster Ordnung der Form
\begin{equation}
  \dot{\vec{\mathbf{x}}}=\mathbf{A}\vec{\mathbf{x}}+\vec{\mathbf{b}}
  \label{eq:InhomSystem}
\end{equation}
Eine Fundamentallösung von (\ref{eq:InhomSystem}) in der Form einer
Matrixfunktion $\boldsymbol{\Phi}(t)$ gehorcht der Gleichung 
\begin{equation}
  \dot{\boldsymbol{\Phi}}(t)=\mathbf{A}\boldsymbol{\Phi}(t)
  \label{eq:Fundamentalmatrix}
\end{equation}
$\boldsymbol{\Phi}(t)=e^{At}$ ist eine Fundamentalmatrix, welche die Bedingung
$\boldsymbol{\Phi}(0)=\mathbbm{1}$ erfüllt. Für (\ref{eq:InhomSystem}) haben wir
$\boldsymbol{\Phi}(t)=e^{At}\mathbf{C}$, mit einer beliebigen nichtsingulären
Matrix $\mathbf{C}$.  Wenn aber mit $\boldsymbol{\Phi(t)}$ eine Fundamentalmatrix
gefunden ist, dann ist es leicht ein System wie (\ref{eq:InhomSystem}) mit den
Anfangsbedingungen $\mathbf{x}(0)=\mathbf{x}_0$ zu lösen
\begin{equation}
  \mathbf{x}(t)=\boldsymbol{\Phi}(t)\boldsymbol{\Phi}^{-1}(0)\mathbf{x}_0
  +\int\limits_0^t\boldsymbol{\Phi}(t)\boldsymbol{\Phi}^{-1}(\tau)\vec{\mathbf{b}}(\tau)d\tau
  \label{eq:LoesInhom}
\end{equation}
Setzen wir hier $\boldsymbol{\Phi}(t)=e^{\mathbf{A}t}$ ein, dann erhalten wir
\[ \mathbf{x}(t)=e^{\mathbf{A}t}\mathbf{x}_0
   +e^{\mathbf{A}t}\int\limits_0^te^{-\mathbf{A}\tau}\;\vec{\mathbf{b}}(\tau)d\tau\]

   \begin{example}{Getriebener harmonischer Oszillator}
     \[\ddot{x}(t)+x(t)=f(t)\]
     kann als inhomogenes System erster Ordnung geschrieben werden. Wir setzen
     $y_1(t)=x(t)$ und $y_2(t)=\dot{x}(t)$
     \begin{align*}
       \dot{y}_1(t)&=y_2(t)\\
       \dot{y}_2(t)&=-y_1(t)+f(t)
     \end{align*}
     In diesem Fall ist
     \[\mathbf{A}=
       \begin{pmatrix}
	 0&1\\ -1&0
       \end{pmatrix}\quad
       \mbox{ und }\quad
       \vec{\mathbf{b}}(t)=\begin{pmatrix}0\\ f(t)\end{pmatrix}
     \]
     Wir wissen, dass
     \[e^{\mathbf{A}t}=
       \begin{pmatrix}
	 \cos(t)&-\sin(t)\\ \sin(t)&\cos(t)
       \end{pmatrix}
       =R(t)
     \]
     eine Rotationsmatrix ist und es ist
     \[e^{-\mathbf{A}t}=
       \begin{pmatrix}
         \cos(t)&\sin(t)\\-\sin(t)&\cos(t)
       \end{pmatrix}
       =R(-t)
     \]
     Wir erhalten als Lösung
     \[\mathbf{y}(t)=R(t)\mathbf{y}_0 + 
       R(t)\int\limits_0^t\begin{pmatrix}f(\tau)\sin(\tau)\\f(\tau)\cos(\tau)\end{pmatrix}d\tau
     \]
     Das ursprüngliche Oszillatorsystem hat also die Lösung $x(t)=y_1(t)$
     \[ x(t)=x_0\cos(t)-v_0\sin(t)+\int\limits_0^tf(\tau)\sin(\tau -t)d\tau \]
   \end{example}