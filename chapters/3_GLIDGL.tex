\chapter{Gewöhnliche lineare inhomogene Gleichung}
\Comment{3. Vorlesung}
\section{Variation der Konstanten}
Diese Methode lässt sich auch bei variablen Koeffizienten anwenden. Die Lösung der homogenen Gleichung hierfür, soll allerdings erst später besprochen werden. Daher gehen wir von einer Differentialgleichung der Form (\ref{eq:DGLOrdnungN})
aus. Die Idee hinter der Methode ist die gefundenen Fundamentallösungen zu
superponieren, die benutzten Koeffizienten aber als noch zu bestimmende
Funktionen anzusetzen. Wir setzen also folgendes an
\begin{equation}
  y(t)=c_1(t)y_1(t)+c_2(t)y_2(t)+\cdots+c_n(t)y_n(t)
  \label{eq:AnsatzVariationKonstanten}
\end{equation}
Nun brauchen wir natürlich auch $n$ Bedingungen um die Funktionen $c_i(t)$ zu
bestimmen. Wir nehmen an dass die $c_i(t)$ mindestens einmal differenzierbar seien.
Die Lösung $y(t)$ hingegen muss n-mal differenzierbar sein. Dann legen wir folgendes fest für die $c_i(t)$
\begin{align}
  \dot{y}(t)=&c_1(t)\dot{y}_1(t)+c_2(t)\dot{y}_2(t)+\cdots+c_n(t)\dot{y}_n(t)\nonumber\\
  &+\underbrace{\dot{c}_1(t)y_1(t)+\dot{c}_2(t)y_2(t)+\cdots+\dot{c}_n(t)y_n(t)}_{=0}\nonumber\\
  \ddot{y}(t)=&c_1(t)\ddot{y}_1(t)+c_2(t)\ddot{y}_2(t)+\cdots+c_n(t)\ddot{y}_n(t)\nonumber\\
  &+\underbrace{\dot{c}_1(t)\dot{y}_1(t)+\dot{c}_2(t)\dot{y}_2(t)+\cdots
               +\dot{c}_n(t)\dot{y}_n(t)}_{=0}\nonumber\\
	       &\cdots\cdots\cdots\nonumber\\
	       y^{(n)}(t)=&c_1(t)y^{(n)}_1(t)+c_2(t)y^{(n)}_2(t)+\cdots+c_n(t)y^{(n)}_n(t)\nonumber\\
  &+\underbrace{\dot{c}_1(t)y^{(n-1)}_1(t)+\dot{c}_2(t)y^{(n-1)}_2(t)+\cdots
  +\dot{c}_n(t)y^{(n-1)}_n(t)}_{=g(t)}\label{eq:Diffbarkeitci}
\end{align}
Gegeben, dass diese Bedingungen erfüllt sind, so ist
(\ref{eq:AnsatzVariationKonstanten}) Lösung der Differentialgleichung. Dies
können wir sofort erkennen, denn wenn wir die so berechneten Ableitungen aus
(\ref{eq:Diffbarkeitci}) in die Differentialgleichung einsetzen, ist diese
erfüllt.

Für die $\dot{c}_k(t)$ dagegen haben wir das Gleichungssystem
\begin{align}
  \dot{c}_1(t)y_1(t)&+\dot{c}_2(t)y_2(t)+\cdots+\dot{c}_n(t)y_n(t)=0\nonumber\\
	       &\cdots\cdots\cdots\nonumber\\
  \dot{c}_1(t)y^{(n-1)}_1(t)&+\dot{c}_2(t)y^{(n-1)}_2(t)+\cdots
  +\dot{c}_n(t)y^{(n-1)}_n(t)=g(t)
	       \label{eq:SystemCs}
\end{align}
Die Koeffizientendeterminante von (\ref{eq:SystemCs}) ist die
Wronskideterminante von der wir in (\ref{eq:WronskianGeneral}) gezeigt haben,
dass sie von null verschieden ist. Damit ist sichergestellt, dass
(\ref{eq:SystemCs}) eine Lösung hat. Es bleibt die Aufgabe die $c_i(t)$ zu berechnen.
%
\begin{example}{Variation der Konstanten}
  \[\ddot{y}(t)+y(t)=\frac{1}{\cos(t)}\]
  %TODO die homogene Lösungen sind ?
  Die allgemeine Lösung ist $y(t)=c_1\cos(t)+c_2\sin(t)$ und wir machen für die
  spezielle Lösung den Ansatz
  \[y(t)=c_1(t)\cos(t)+c_2(t)\sin(t)\]
  und erhalten das lineare Gleichungssystem
  \begin{align*}
    \dot{c}_1(t)\cos(t)+\dot{c}_2(t)\sin(t)&=0\\
    -\dot{c}_1(t)\sin(t)+\dot{c}_2(t)\cos(t)&=\frac{1}{\cos(t)}
  \end{align*}
  Die spezielle Lösung ist damit
  \[y(t)=\cos(t)\ln(|\cos(t)|)+t\sin(t)\]
\end{example}
%
\section{Lösung der inhomogenen Gleichung: Operatormethode}
Für die Ableitung benutzen wir als Abkürzung eine Operatorschreibweise der Form
$\frac{dy(t)}{dt}=D[y(t)]$. Entsprechend soll $D^{-1}[y(t)]=\int y(t)dt$
verwendet werden. Wir wissen ja aus dem Haupsatz der Differential- und
Integralrechnung, dass die Ableitung die Umkehrung des Integrals ist. Für einen
Operator $D$ schreiben wir dessen Inverse als $D^{-1}$. Ein r-faches
Hintereinanderausführen eines Operators können wir also mit $D^r$ schreiben.
\begin{note}{}
  Mit dieser Schreibweise wird uns klar, dass z.B. die Exponentialfunktion der
  Ableitung
  \[e^\frac{d}{dt}=e^D=1+D+\frac{D^2}{2}+\dots=\sum_{\nu=0}^\infty\frac{D^\nu}{\nu!}\]
  bedeutet. 
  
  Merke: Funktionen von Operatoren sind über ihre Reihenentwicklung
  definiert.

  Eine offensichtliche Anwendung sehen wir mit der Taylorentwicklung von
  $y(t+\tau)$ um den Zeitpunkt $t$
  \[y(t+\tau)=y(t)+\left.\frac{\tau}{1!}\dot{y}(s)\right|_{s=t}+\left.\frac{\tau^2}{2!}\ddot{y}(s)\right|_{s=t}
	+\dots=\sum\limits_{\nu=0}^{\infty}\left.\frac{\tau^\nu}{\nu!}\frac{d^\nu}{ds^\nu}y(s)\right|_{s=t}=
	e^{\tau\frac{d}{dt}}y(t)\]
  Wir nennen $e^{\tau\frac{d}{dt}}=e^{\tau D}$ den Verschiebungs- oder
  Shiftoperator.
\end{note}
Wir betrachten die folgende Differentialgleichung erster Ordnung mit konstanten
Koeffizienten 
\begin{equation}
  \dot{y}(t)-\lambda y(t)=g(t)\mbox{ oder in Operatorform }(D-\lambda)y(t)=g(t)
  \label{eq:RelaxatorOperatorform}
\end{equation}
Wir wissen, dass (\ref{eq:RelaxatorOperatorform}) als homogene Lösung
$y(t)=e^{\lambda t}$ hat, also versuchen wir es mit dem
Produktansatz\footnote[1]{Eingesetzt ergibt dies $D[e^{\lambda
t}D^{-1}[e^{-\lambda t}g(t)]]-\lambda e^{\lambda t}D^{-1}[e^{-\lambda
t}g(t)]=g(t)$. Wenn wir die Operation $D$ ausführen - Achtung Produktregel -
und $DD^{-1}=1$ beachten, dann erhalten wir $g(t)=g(t)$}
\begin{equation}
  y(t)=e^{\lambda t}\cdot D^{-1}[e^{-\lambda t}g(t)]
  \label{eq:Produktansatz}
\end{equation}
Wir verifizieren das durch einsetzen in (\ref{eq:RelaxatorOperatorform}). Nun
wenden wir das auf die Differentialgleichung $(D-\lambda)^2[y(t)]=g(t)$ an und
machen hierfür den entsprechenden Produktansatz
\begin{equation}
  y(t)=e^{\lambda t}\cdot D^{-2}[e^{-\lambda t}g(t)]
  \label{eq:Produktansatz2Ordnung}
\end{equation}
und zeigen wieder, dass (\ref{eq:Produktansatz2Ordnung}) die
Differentialgleichung erfüllt. nun stellt sich die Frage nach der
Verallgemeinerung, d.h. was geschieht wenn die Differentialgleichung 
\begin{equation}
  \ddot{y}(t)-\alpha\dot{y}(t)+\beta y(t)=g(t)
  \label{eq:DGL2Ordnung2Koeffizienten}
\end{equation}
vorliegt, woebei wir $\alpha=\lambda_1+\lambda_2$ und $\beta=\lambda_1\lambda_2$
schreiben. Damit haben wir
\begin{equation}
  (D-\lambda_1)(D-\lambda_2)[y(t)]=g(t)
  \label{eq:DGG2KoeffizientenOperatorform}
\end{equation}
Die Gleichung (\ref{eq:DGL2Ordnung2Koeffizienten}) können wir mit $(D-\lambda_2)[y(t)]=y^*(t)$ 
in der Form $(D-\lambda_1)[y^*(t)]=g(t)$ schreiben.

Nun lösen wir das ganze sukzessive auf
\begin{align} 
  y^*(t)&=e^{\lambda_1t}\cdot D^{-1}\left[e^{-\lambda_1 t}g(t)\right]\nonumber\\
  y(t)  &=e^{\lambda_2t}\cdot D^{-1}\left[e^{-\lambda_2 t}y^*(t)\right]=\nonumber\\
        &=e^{\lambda_2t}\cdot D^{-1}\left[e^{(\lambda_1-\lambda_2) t}\cdot
      D^{-1}\left[e^{-\lambda_1 t}g(t)\right]\right]
  \label{eq:DGL2OrdnungSolu}
\end{align}
Damit ist auch der allgemeine Fall für ein Polynom beliebiger Ordnung 
\[ P(D)[y(t)]=(D-\lambda_1)^{r_1}\cdot(D-\lambda_2)^{r_2}\cdots 
(D-\lambda_n)^{r_n}[y(t)]\]
zugänglich, nämlich
\begin{equation}
  y(t)=e^{\lambda_s t}\cdot D^{-r_s}[e^{(\lambda_{s-1}-\lambda_s)t}\cdot D^{-r_{s-1}}
  [\dots e^{(\lambda_1-\lambda_2)t}D^{-r_1}[e^{-\lambda_1t}g(t)]\dots]]
  \label{eq:SoluInhomogenOperator}
\end{equation}
\begin{example}{Ganzrationale Inhomogenität}
  \[\dddot{y}(t)-3\dot{y}(t)-2y(t)=4t^2-2\]
  Hat die characteristische Gleichung
  \[(\lambda^3-3\lambda-2)=(\lambda-2)(\lambda+1)^2=0\]
  Die allgemeine Lösung der homogenen Differentialgleichung lautet somit
  \[y(t)=c_1e^{2t}+c_2e^{-t}+c_3te^{-t}\]
  Mit (\ref{eq:SoluInhomogenOperator}) erhalten wir
  \[
    y(t)=e^{-t}\cdot D^{-2}[e^{3t}\cdot D^{-^1}[e^{-2t}(4t^2-2)]]=-2t^2+6t-8
  \]
  Damit ist die allgemeine Lösung gegeben durch
  \[ y(t)=c_1e^{2t}+c_2e^{-t}+c_3te^{-t}-2t^2+6t-8 \]
  Die $c_i$ werden durch die Anfangsbedingungen festgelegt.
\end{example}
Hieraus leiten wir eine allgemeine herangehensweise bei Polynomen als
Inhomogenitäten ab.
\begin{note}{Spezielle Lösung bei speziellen Inhomogenitäten}
  \begin{enumerate}
    \item Ist die Inhomogenität gegeben als ein Polynom $n$-ter Ordnung, so setzt
      man die spezielle Lösung der Differentialgleichung als allgemeines
      Polynom $n$-ter Ordnung an und bestimmt die Polynomialkoeffizienten durch
      Einsetzen in die inhomogene Differentialgleichung und
      Koeffizientenvergleich.  Wir haben 2 Fälle zu betrachten
    \item Ist $g(t)=e^{\mu t}$ und $\mu\ne\lambda_i$ dann erhalten wir durch
      einsetzen in (\ref{eq:SoluInhomogenOperator}) 
  \[ y(t)=\frac{e^{\mu t}}{(\mu-\lambda_1)^{r_1}
            \cdots(\mu-\lambda_s)^{r_s}}\]
	  \item Liegt hingegen Resonanz vor, z.B.  $\mu=\lambda_s$, dann schreiben wir (\ref{eq:SoluInhomogenOperator})
	    \[y(t)=e^{\lambda_s t}\cdot D^{-r_s}
	    \left[\frac{e^{(\mu-\lambda_s)\cdot t}}{(\mu-\lambda_1)^{r_1}\dots(\mu-\lambda_{s-1})^{r_{s-1}}}\right]\]
	    Beachte, die Reihenfolge der Operatoren $(D-\lambda_i)^{r_i}$
	    kann beliebig vertauscht werden und wir erhalten im Resonanzfall
	    \[y(t)=e^{\mu t}\cdot D^{-r_s}
	      \left[\frac{1}{(\mu-\lambda_1)^{r_1}\dots(\mu-\lambda_{s-1})^{r_{s-1}}}\right]\]
	    Was sich wiederum integrieren lässt zu
	\[ y(t)=\frac{t^{r_s}e^{\mu t}}{(\mu-\lambda_1)^{r_1} \cdots(\mu-\lambda_{s-1})^{r_{s-1}}}
	\cdot\frac{1}{r_s!}\]
  \end{enumerate}
\end{note}
Wir fassen die allgemeine Vorgehensweise im Diagramm in Abbildung
\ref{fig:FlowchartSolu} zusammen.
%
\begin{figure}[htb]
  \centering
  \tikzstyle{decision} = [diamond, draw, 
    text width=4.5em, text badly centered, node distance=3cm, inner sep=0pt]
  \tikzstyle{block} = [rectangle, draw, 
    text width=8em, text centered, rounded corners, minimum height=4em]
  \tikzstyle{line} = [draw, -latex']
  \tikzstyle{cloud} = [draw, ellipse, node distance=3cm,
    minimum height=2em]
\begin{tikzpicture}[node distance = 2cm, auto]
    % Place nodes
    \node [block] (init) {DGL-Löser};
    \node [block, left of=init, node distance=4.5cm] (dgl) {Differential- gleichung};
    \node [block, right of=init, node distance=4.5cm] (ics) {Anfangsbe- dingungen};
    \node [decision, below of=init] (hominhom) {ist es eine homogene DGL?};
    \node [block, right of=hominhom, node distance=4.5cm] (partsolu) {Suche spezielle L\"osung};
    \node [block, below of=hominhom, node distance=3.5cm] (homsolu) {Suche homogene L\"osung(en)};
    \node [block, below of=homsolu, node distance=2.5cm] (hometinhom) {Addiere
           homogene und spezielle L\"osung(en)};
    \node [block, below of=hometinhom, node distance=2.5cm] (stop) {Erfülle Anfangsbedingungen};
         % Draw edges
    \path [line] (init) -- (hominhom);
    \path [line] (hominhom) -- node {immer} (homsolu);
    \path [line] (hominhom) -- node {nein} (partsolu);
    \path [line] (homsolu) -- (hometinhom);
    \path [line] (hometinhom) -- (stop);
    \path [line] (dgl) -- (init);
    \path [line] (ics) -- (init);
    \path [line] (partsolu) |- (hometinhom);
\end{tikzpicture}
\caption{Flussdiagramm zur Lösung inhomogener linearer Differentialgleichungen.}
  \label{fig:FlowchartSolu}
\end{figure}
Zur Erfüllung der Randbedingungen gibt es zwei Vorgehensweisen, die wir im
folgenden diskutieren wollen.

In der ersten Vorgehensweise überlagern wir die gefundene spezielle Lösung mit
der homogenen Lösung, die ja noch zu wählende Konstanten beinhaltet
$y(t)=y_{hom}(t)+y_{par}(t)$. Dann setzen wir diese Lösung in die
Randbedingungen ein und bestimmen so die Konstanten. Dies ist die sichere
Variante.

Für die zweite Vorgehensweise erinnern wir uns an die Variation der Konstanten.
Dort hatten wir für eine inhomogene Gleichung erster Ordnung
$(D-\lambda)y(t)=f(t)$ den Ansatz $y_{par}(t)=c(t)\cdot e^{\lambda t}$ gemacht.
Die Besitummungsgleichung für $c(t)$ lautete $\dot c(t)=e^{\lambda t}\cdot
f(t)$. Nun integrieren wir über t in den Grenzen von $0$ bis $t$. Damit
erhalten wir $c(t)-c(0)=\int_0^te^{\lambda t'}\cdot f(t')dt'$. Die Konstante
$c(0)$ ist unbestimmt. Wir wählen $c(0)=0$. Die Integration in den Grenzen
wenden wir auch auf (\ref{eq:SoluInhomogenOperator}) an. Damit wird
(\ref{eq:SoluInhomogenOperator}) zu
\begin{equation}
  y_p(t)=e^{\lambda_n t}
\int\limits_0^t e^{(\lambda_{n-1}-\lambda_n) t'}
\int\limits_0^{t'} e^{(\lambda_{n-2}-\lambda_{n-1}) t''}\dots
\int\limits_0^{r} e^{(\lambda_{1}-\lambda_{2}p) p}\
\int\limits_0^{p} e^{-\lambda_{1}q}f(q)dqdp\dots dt''dt'
  \label{eq:SoluInhomogenOperator0}
\end{equation}
Wie man leicht sieht, gilt $y_p(0)=0$, $\dot y_p(0)=0$,\ldots,
$y^{(n-1)}_p(0)=0$ (wir zeigen dies in der Anmerkung unten). Damit bleibt die
Aufgabe, die $c_i$ der homogenen Lösung so zu bestimmen, dass die homogene
Lösung die Anfangsbedingungen erfüllt.
\begin{note}{Die Ableitungen von $y_p(t)$ aus (\ref{eq:SoluInhomogenOperator0}) bei $t=0$}
Wir leiten (\ref{eq:SoluInhomogenOperator0}) einmal nach $t$ ab, dann erhalten
wir (Kettenregel beachten)
\begin{align} \dot y_p(t)=\lambda_n y_p(t)+
e^{\lambda_n t}\frac{d}{dt}&
\int\limits_0^t e^{(\lambda_{n-1}-\lambda_n) t'}
\int\limits_0^{t'} e^{(\lambda_{n-2}-\lambda_{n-1}) t''}\dots\\
&\dots\int\limits_0^{r} e^{(\lambda_{1}-\lambda_{2}p) p}\
\int\limits_0^{p} e^{-\lambda_{1}q}f(q)dqdp\dots dt''dt'\nonumber
\end{align}
Unter dem Integral über $dt'$ steht eine Funktion $g(t')$, n'est-ce pas? Nämlich
\[g(t')=e^{(\lambda_{n-1}-\lambda_n) t'}
\int\limits_0^{t'} e^{(\lambda_{n-2}-\lambda_{n-1}) t''}\dots
\int\limits_0^{r} e^{(\lambda_{1}-\lambda_{2}p) p}\
\int\limits_0^{p} e^{-\lambda_{1}q}f(q)dqdp\dots dt''
\]
Also haben wir
\[ \dot y_p(t)=\lambda_n y_p(t)+e^{\lambda_n t}\frac{d}{dt}
\int\limits_0^t g(t')dt'
\]
Davon wissen wir aber, dass gilt
\[ \lambda_n y_p(t)+e^{\lambda_n t} g(t)
\]
Es ist aber $y_p(0)=0$ und, wie wir leicht sehen, auch $g(0)=0$. Der geneigte Leser möge letzte Aussage beweisen. Also ist auch $\dot y_p(0)=0$. Das geht weiter bis zur Ableitung der Ordnung $n-1$.
\end{note}
%
\section{Randwertprobleme}
Gegeben eine lineare Differentialgleichung mit konstanten Koeffizienten 2. Ordnung
\begin{equation}
    \frac{d^2y(x)}{dx^2}+a\frac{dy(x)}{dx}+by(x)=g(x)
    \label{eq:LinDGL2Ordnung}
\end{equation}
Wir machen einen Exponentialansatz $y(x)=e^{\lambda x}$ und erhalten für die homogene Differentialgleichung (\ref{eq:LinDGL2Ordnung})
\begin{equation}
    \lambda^2+a\lambda+b=0
    \label{eq:CharGL}
\end{equation}
als charackteristische Gleichung. Hat (\ref{eq:CharGL}) zwei Lösungen, so ergibt dies auch zwei Lösungen $y_1(x)=e^{\lambda_1} x$ und $y_2(x)=e^{\lambda_2}x$. Die zwei Lösungen überlagern wir um eine Lösung von (\ref{eq:LinDGL2Ordnung}) zu erhalten
\[ y(x)=c_1 y_1(x)+c_2 y_2(x) \]

Nun sei $g(x)=0$ und die beiden Randbedingungen $y(0)=\alpha$ und $y(L)=\beta$ seien gegeben. Damit erhalten wir foldendes Gleichungssystem, das wir erfüllen müssen
\begin{equation}
    \begin{pmatrix}y_1(0)&y_2(0)\\ y_1(L)&y_2(L)\end{pmatrix}
    \begin{pmatrix}c_1\\ c_2\end{pmatrix} = 
     \begin{pmatrix}\alpha \\ \beta\end{pmatrix}
\label{eq:Randbedingungen}
\end{equation}
Aus (\ref{eq:Randbedingungen}) berechnen wir die Lösungen für $c_1$ und $c_2$, wobei diese Gleichung nicht notwendigerweise eine Lösung hat, dies ist zuerst zu zeigen. Sollte (\ref{eq:Randbedingungen}) nicht lösbar sein, so haben wir Randbedingungen gefordert, die durch das Gleichungssysten nicht efüllt werden können.

In Falle $g\ne 0$ gehen wir vor, wie wir es bei den Anfangswertproblemen auch getan haben. Wir überlagern die homogene und die partikuläre Lösung $y(x)=y_h(x)+y_p(x)$ und bekommen ein Gleichungssystem, das nun auch die Funktion $g(x)$ enthält, nämlich durch (\ref{eq:DGL2OrdnungSolu}) eben in der Form
\[ y_p(x)=e^{\lambda_2 x}\int\limits_0^x e^{(\lambda1-\lambda_2)\eta}\int\limits_0^{\eta}e^{-\lambda_1 \xi}g(\xi)d\xi d\eta\]
Damit wird (\ref{eq:Randbedingungen}) zu
\begin{equation}
    \begin{pmatrix}y_1(0)&y_2(0)\\ y_1(L)&y_2(L)\end{pmatrix}
    \begin{pmatrix}c_1\\ c_2\end{pmatrix} = 
     \begin{pmatrix}\alpha+y_p(0) \\ \beta+y_p(L)\end{pmatrix}
\label{eq:RandBDInhom}
\end{equation}