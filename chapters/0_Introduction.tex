\chapter[Erinnerung an die \dots]{Erinnerung an die Integral- und
Differentialrechnung und die Lineare Algebra}
In diesem Kapitel wollen wir an die in den Grundvorlesungen der Mathematik für
Ingenieurinnen und Informatiker behandelten Themen erinnern.
\section{Sätze der Differential- und Integralrechnung}
\begin{satz}{Hauptsatz der Differential- und Integralrechnung\label{theo:HSIntegralDiff}}
Wenn für eine stetige Funktion $f(x)$ das Integral
$F(x)=\int\limits_{x_0}^{x}f(\xi)d\xi$ existiert, dann ist
\[
  \frac{dF(x)}{dx}=\frac{d}{dx}\int\limits_{x_0}^{x}f(\xi)d\xi=f(x)
\]
\label{theo:HDI}
\end{satz}
Satz \ref{theo:HDI} in Worten: ``Die Ableitung des Integrals ist gleich der
Funktion im Integranden an der oberen Integrationsgrenze''.
\begin{example}{Ableitung eines Integrals}
\[
 \frac{d F(x)}{dx}=\frac{d}{dx}\int\limits_{x_0}^{x}e^{-\xi}d\xi=
 \frac{d}{dx} \left(-e^{-x}+e^{-x_0}\right)=e^{-x}
\]
\end{example}
\begin{satz}{Mittelwertsatz der Integralrechnung}
  Gegeben eine Funktion $f(x)$, die auf dem Interval $[a,b]$ stetig ist, sowie
  eine Funktion $g(x)$ für die im Intervall $[a,b]$ entweder $g(x)\ge0$ oder
  $g(x)\le0$ gelte. Dann gibt es mindestens eine Stelle $\xi\in[a,b]$, so dass
  mit deren entsprechendem Funktionswert 
  \begin{equation} 
  \int\limits_a^bf(x)g(x)dx=f(\xi)\int\limits_a^bg(x)dx 
  \label{eq:MittelwertInt2} 
  \end{equation}
  gilt.   
\end{satz}
Wenn wir $g(x)=1 $ setzen, dann wird dies gemeinhin als erster
Mittelwertsatz der Integralrechnung bezeichnet. Es gilt also:
\begin{equation} 
  \int\limits_a^bf(x)dx=f(\xi)(b-a)
  \label{eq:MittelwertInt1}
\end{equation}
%\begin{example}{Anwendung von Satz \ref{eq:MittelwertInt2}}
%?
%\end{example}
\begin{satz}{Mittelwertsatz der Differentialrechnung}
  Gegeben 2 Funktionen $f(x)$ und $g(x)$, die beide auf dem Interval $[a,b]$
  stetig und differenzierbar seien. Dann existiert ein $x_0\in(a,b)$, so dass   
  \begin{equation}
    f'(x_0)\left(g(b)-g(a)\right)= g'(x_0)\left(f(b)-f(a)\right)
  \label{eq:MittelwertDiff2}
\end{equation}
\end{satz}
Wenn wir $g'(x)=1 $ setzen, dann wird dies gemeinhin als erster
Mittelwertsatz der Differetialrechnung bezeichnet. Es gilt also:
\begin{equation}
  f'(x_0)=\frac{f(b)-f(a)}{b-a}
  \label{eq:MittelwertDiff1}
\end{equation}
\section{Taylorreihe}
%{\red Das könnte ein Beispiel für den Beweis eines Satzes sein}
Die Taylorreihe ist zweifelsohne eines der wichtigen Instrumente. Sie dient zur
Abschätzung des Verhaltens von Funktionen und findet in numerischen
Methoden Anwendung, wie z.B.\ den Finiten Differenzen.
\begin{satz}{Der Taylorsche Satz\label{theo:Taylor}} Sei $f$ eine Funktion, die
  eine $(n+1)$-te stetige Ableitung auf einem Intervall $J$ besitze. Es seien
  $a,b\in J$. Dann ist 
\[
    f(b)=f(a)+\frac{b-a}{1!}f'(a)+\cdots +\frac{(b-a)^{n}}{(n)!}f^{(n)}(a)+R_n
\]
  mit
\[
    R_n=\int\limits_a^b\frac{(b-s)^{n}}{n!}f^{(n+1)}(s)ds
\]
Des weiteren merken wir an, dass eine Zahl $c\in[a,b]$ existiert, so dass
\[
  R_n=\frac{(b-a)^{n+1}}{(n+1)!}f^{(n+1)}(c)
\]
gilt.
\end{satz}
{\bf Beweis:} Wir wenden Satz 1 an und integrieren partiell
\begin{align}
  f(b)=&f(a)+\int\limits_a^bf'(s)ds=f(a)+\int\limits_a^b\frac{(b-s)^0}{0!}f'(s)ds\nonumber\\
  =&f(a)-\left.\frac{(b-s)^1}{1!}f'(s)\right|_a^b+\int\limits_a^b\frac{(b-s)^1}{1!}f''(s)ds=\dots
  \label{eq:TaylorProof1}
\end{align}
Vollständige Induktion liefert den Schluss des Beweises.
\begin{example}{Taylorreihenentwicklung}
Man entwickle $f(x)=e^{-x}$ um $x=0$
  \[ e^{-x}=1-x+\frac{1}{2!}x^2-\frac{1}{3!}x^3+R_3 \]
\end{example}
\section{Lineare Algebra}
In der Mathematik treffen wir auf verschiedene Arten von Objekten, die
untereinander addiert und mit Zahlen multipliziert werden können. Eine
Ansammlung von Objekten nennen wir eine Menge. Ein Beispiel ist die Menge von
Vektoren ein und derselben Dimension. Wir wollen für all die
Spezialfälle eine allgemeine Definition geben. 
\subsection{Der Vektorraum}
Ein {\bf Vektorraum} V ist eine Menge von Objekten, die addiert und mit Zahlen
multipliziert werden können und zwar so, dass die Summe zweier Elemente aus V
wieder ein Element aus V ergibt, das Produkt eines Elementes aus V mit einer Zahl
ebenfalls wieder in V ist und folgende Eigenschaften erfüllt sind:.
\begin{description}
  \item[{\bf VR 1.}] Gegeben $u,v,w\in V$. Es soll Assoziativität gelten: $(u+v)+w = u+(v+w)$
  \item[{\bf VR 2.}] Es gibt ein Element 0 in V, so dass gilt $0+u = u+0 =u$
    für alle $u\in V$. Wir nennen dieses Element den Nullvektor.
  \item[{\bf VR 3.}] Für jedes $u\in V$ gibt es ein Element $(-1)u\in V$ für das $u+(-1)u=0$
    gilt. D.h. jedes Element hat ein Inverses.
  \item[{\bf VR 4.}] Für alle $u,v\in V$ soll Kommuntativität gelten $u+v = v+u$
  \item[{\bf VR 5.}] Für eine Zahl $c$ gilt Distributivität $c(u+v)=cu+cv$ 
  \item[{\bf VR 6.}] Für zwei Zahlen $a$, $b$ gilt Assoziativität bezüglich der
    Addition $(a+b)u=au+bu$
  \item[{\bf VR 7.}] Für zwei Zahlen $a$, $b$ gilt Assoziativität bezüglich der
    Multiplikation $(ab)u=a(bu)$
  \item[{\bf VR 8.}] Für alle $u\in V$ gilt $1\cdot u=u$
\end{description}
Als einen Unterraum $W$ von $V$ bezeichnen wir einen Raum für den gilt
\begin{enumerate}
  \item $\forall u,v\in W$ ist auch $u+v\in W$
  \item $\forall u\in W$ und $c\in\mathbb{R}$ ist auch $cu\in W$
  \item $0\in V$ ist auch Element von $W$ 
\end{enumerate}
\begin{example}{Beispiele für Vektorräume}
  \begin{itemize}
    \item Der dreidimensionale Euklidische Raum $\mathbf{R}^3$ ist ein Vektorraum.
    \item Die Menge der Polynome kleiner gleich vierten Grades 
      \[p_4(x)=a_0+a_1x+a_2x^2+a_3x^3+a_4x^4\]
      ist ein Vektorraum.
  \end{itemize}
  Man verifiziere dies anhand der oben gegebenen Definition.
\end{example}
 \subsection{Lineare Unabhängigkeit, Basis}
Eine Linearkombination von Vektoren $(v_1,\dots,v_m)\in V$ schreiben wir als
$a_1v_1+\dots+a_mv_m$, wobei $a_1,\dots,a_m\in\mathbb{R}$ oder auch
$a_1,\dots,a_m\in\mathbb{C}$. Alle möglichen Linearkombinationen der $v_i$
nennen wir die lineare Hülle der $(v_1,\dots,v_m)\in V$ und schreiben dafür
$span(v_1,\dots,v_m)$. Die $m$ Vektoren spannen einen Unteraum $W\subset V$
auf.
\begin{example}{Lineare Hülle}
Es ist
\[(7,2,9)=2(2,1,3)+3(1,0,1). \]
Der Vektor $(7,2,9)$ ist also eine Linearkombination von $(2,1,3)$ und
$(1,0,1)$.  Daher sagen wir auch $(7,2,9)\in span\left( (2,1,3),(1,0,1)
\right)$ und damit nennen wir diesen Vektor auch linear abhängig.
\end{example}
Wir nennen Vektoren $(v_1,\dots,v_m)\in V$ linear unabhängig, wenn keiner der
$m$ Vektoren als Linearkombination der restlichen $m-1$ Vektoren geschrieben
werden kann. Eine Basis des Vektorraums $V$ ist gegeben durch die maximale
Anzahl linear unabhängiger Vektoren, die ganz $V$ aufspannen. Diese muss nicht
eindeutig sein.
\begin{example}{Basen}
  \begin{itemize}
    \item $(1,0,0)$, $(0,1,0)$ und $(0,0,1)$ spannen der $\mathbb{R}^3$ auf. 
    \item Genauso tut das aber auch die Basis
      $(1/\sqrt{2},-1/\sqrt{2},0)$,$(-1/\sqrt{2},1/\sqrt{2},0)$ und $(0,0,1)$.
    \item Die Monome $\{1,x,x^2,x^3,x^4\}$ bilden die Basis der Vektorraums aus
      der Menge der Polynome vierten Grades und kleiner.
  \end{itemize}
\end{example}
\subsection{Der Rang einer Matrix}
Gegeben die Matrix
\begin{equation}
  \mathbf{A}=
  \begin{pmatrix}
    a_{11}&a_{12}&\;\cdots\;&a_{1n}\\
    a_{21}&a_{22}&\;\cdots\;&a_{2n}\\
    & &\;\cdots\;&\\
    a_{m1}&a_{m2}&\;\cdots\;&a_{mn}\\
  \end{pmatrix}
  \label{eq:Matrixmn}
\end{equation}
\begin{note}{}
  Vektoren und Matrizen schreiben wir in Fettdruck. Wir bezeichnen der Rang
  einer Matrix $\mathbf{A}$ mit $\text{Rg}(\mathbf{A})$.
\end{note}
Durch Streichen von Zeilen oder Spalten erhalten wir quadratische $s\times s$
Untermatrizen. Gehen wir davon aus, dass $A$ nicht die Nullmatrix ist, dann finden
wir sicherlich unter den $s\times s$ Untermatrizen solche, deren Determinanten
von Null verschieden sind. Die Maximale Zahl $r$ an Reihen, bzw.  Spalten, der
von Null verschiedenen Determinanten, die bei dieser Operation entstehen,
nennen wir den Rang der Matrix $\text{Rg}(\mathbf{A})$.
\begin{example}{Rang einer $5\times7$-Matrix}
  Man bestimme den Rang der Matrix
  \[
    A=\begin{pmatrix}
    7&1&0&2&-1&4&5\\
    1&1&2&3& 0&1&2\\
    0&1&-2&1&2&0&1\\
    4&-1&-8&-6&1&1&0\\
    0&1&2&1&4&0&1\\
    \end{pmatrix}
  \]
  Wir wissen aus der linearen Algebra, dass der Wert einer Detrminante
  unverändert bleibt, wenn zu einer Zeile bzw. einer Spalte ein beliebiges
  Vielfaches einer Zeile bzw. einer Spalte addiert wird.

  Dies machen wir uns zunutze und addieren Vielfache der Zeilen der obigen
  Matrix sukzessive solange, bis folgende Matrix entsteht: 
  \[
    \begin{pmatrix}
    1&0&0&0&0&0\\
    0&1&0&0&0&0\\
    0&0&1&-2&-2&1\\
    0&0&0&13&11&-3\\
    \end{pmatrix}
  \]
  Beachte: alle linear abhängigen Zeilen und Spalten können wir streichen. 
  
  Als letzte Umformung erhalten wir
  \[
     \begin{pmatrix}
    1&0&0&0\\
    0&1&0&0\\
    0&0&1&0\\
    0&0&0&1\\
    \end{pmatrix}
  \]
  woraus wir den Rang $r=\text{Rg}(\mathbf{A})=4$ ablesen.
\end{example}
Der Rang einer Matrix $\mathbf{M}$ ist die maximale Anzahl linear unabhängiger
Zeilen bzw. Spalten. Die linear unabhängigen Zeilenvektoren spannen die lineare
Hülle des Zeilenvektorraums auf.
%
\subsection{Lineare Gleichungssysteme (LGS)}
Nach dem Satz von Kronecker-Capelli ist ein LGS 
$\mathbf{A\cdot x}=\mathbf{b}$ genau dann lösbar, wenn der Rang der
Koeffizientenmatrix $\mathbf{A}$ gleich dem Rang der erweiterten
Koeffizientenmatrix $\mathbf{A|b}$ ist. Die erweiterte Koeffizientenmatrix
$\mathbf{A|b}$ bilden wir, indem wir den Spaltenvektor $\mathbf{b}$ an die
Matrix $\mathbf{A}$ anfügen. Wir schreiben dies $\mathbf{\{A|b\}}$.
\begin{example}{Lösbarkeit eines LGS}
  \begin{itemize}
    \item
      \[
	\underbrace{\begin{pmatrix}1&1\\2&2\end{pmatrix}}_{\mathbf{A}}
        \underbrace{\begin{pmatrix}x_1\\x_2\end{pmatrix}}_{\mathbf{x}}=
        \underbrace{\begin{pmatrix}2\\4\end{pmatrix}}_{\mathbf{b}}
	\qquad\text{ also schreiben wir }\mathbf{A\cdot x}=\mathbf{b}
      \]
      Es ist $\text{Rg}(\mathbf{A})=1$ und $\text{Rg}(\mathbf{\{A|b\}})=1$,
      also ist das LGS lösbar. Allerdings erhalten wir in diesem Fall eine
      paramterische Lösung und es existieren somit unendlich viele Lösungen.
    \item
       \[
        \underbrace{\begin{pmatrix}1&1\\2&2\end{pmatrix}}_{\mathbf{A}}
        \underbrace{\begin{pmatrix}x_1\\x_2\end{pmatrix}}_{\mathbf{x}}=
        \underbrace{\begin{pmatrix}2\\3\end{pmatrix}}_{\mathbf{c}}
        \qquad\text{ also schreiben wir }\mathbf{A\cdot x}=\mathbf{c}
      \]
      Es ist $\text{Rg}(\mathbf{A})=1$ und $\text{Rg}(\mathbf{\{A|c\}})=2$,
      also ist das LGS nicht lösbar. In diesem Fall sehen wir das auch gleich
      mit bloßem Augen, denn die beiden Gleichungen $x_1+x_2=2$ und
      $x_1+x_2=1.5$ führen auf einen Widerspruch.

      Wohingegen das LGS
    \item
       \[
        \underbrace{\begin{pmatrix}1&1\\1&2\end{pmatrix}}_{\mathbf{D}}
        \underbrace{\begin{pmatrix}x_1\\x_2\end{pmatrix}}_{\mathbf{x}}=
        \underbrace{\begin{pmatrix}2\\3\end{pmatrix}}_{\mathbf{c}}
	\qquad(\text{wir schreiben }\mathbf{D\cdot x}=\mathbf{c})
      \]
      eine eindeutige Lösung hat. Der Nachweis sei dem geneigten Leser überlassen.

      \textbf{Anleitung}: Berechne zunächst zur Übung $\text{Rg}(\mathbf{D})$ und
      $\text{Rg}(\mathbf{\{D|c\}})$. Berechne dann die Lösung.


  \end{itemize}
\end{example}
\subsection{Das Inverse einer Matrix}
Gegeben eine $n\times n$ Matrix $\mathbf{A}$. Gibt es ein Inverses $\mathbf{A}^{-1}$ zu dieser Matrix, so dass $\mathbf{A}^{-1}\cdot\mathbf{A}=\mathds{1}$ die $n\times n$ Einheitsmatrix ergibt? In der Tat gibt es dies, wenn die Matrix $\mathbf{A}$ regulär ist. Eine Matrix ist regulär, wenn Ihre Determinante von Null verschieden ist, sonst heisst die singulär.

Berechnet wir die Inverse folgendermaßen
\begin{equation}\label{eq:matrixinverse}
    \mathbf{A}^{-1}=\frac{\text{adj }\mathbf{A}}{\text{det }\mathbf{A}},
\end{equation}
Wobei die Elemente $A^{ad}_{ij}$ der Adjunkten $\text{adj }\mathbf{A}$ der Matrix $\mathbf{A}$ folgendermaßen definiert sind: Man streiche die Zeile $i$ und die Spalte $j$ der Matrix $\mathbf{A}$ transponiere diese $(n-1)\times(n-1)$-Matrix, berechne deren Determinante und multipliziere diese mit $(-1)^{i+j}$. Zur praktischen Durchführung nehmen wir allerdings den Algorithmus von Gauß und Jordan:

Erweitere die Matrix $\mathbf{A}$ um eine Einheitsmatrix $\mathds{1}$ derselben Dimension
\begin{equation*}
    (\mathbf{A}|\mathds{1})=
    \begin{pmatrix}
    a_{11}&\dots&a_{1n}&1&&0\\
    \vdots&&\vdots&&\ddots&\\
    a_{n1}&\dots&a_{nn}&0&&1
    \end{pmatrix}
\end{equation*}
Nun wenden wir die erlaubten Zeilenoperationen auf die erweiterte Matrix an, sodass aus der linken Hälfte die $n\times n$ Einheitsmatrix $\mathds{1}$ wird, dann erhalten wir in der rechten Hälfte die inverse der Matrix $\mathbf{A}$. Sollte die Inverse nicht existieren, dann merken wir das sofort daran, dass auf der Hauptdiagonalen eine Null resultiert.
\subsection{Eigenwerte, Eigenvektoren, Normalformen}
Die Menge der Eigenwerte und Eigenvektoren einer Matrix nennt man auch ihr Spektrum. Von den drei Matrizen $\mathbf{M}_i$ mit $i=1,2,3$
\[
  \mathbf{M}_1=\begin{pmatrix}
    -9&2&6\\5&0&-3\\-16&4&11
  \end{pmatrix},\quad
  \mathbf{M}_2=\begin{pmatrix}
    2&4&-2\\4&2&-2\\-2&-2&-1
  \end{pmatrix}\text{ und }
  \mathbf{M}_3=\begin{pmatrix}
    6&-6&5\\14&-13&10\\7&-6&4
  \end{pmatrix}
\]
wollen wir das Spektrum berechnen. Es ist $\text{Rg}(\mathbf{M}_i)=3$ (Nachweis!).
\[
  \begin{vmatrix}
    -9-\lambda&2&6\\5&0-\lambda&-3\\-16&4&11-\lambda
  \end{vmatrix}
  =-(\lambda-1)\cdot(\lambda+1)\cdot(\lambda-2)=0
\]
Wir erhalten drei verschiedene Eigenwerte $\lambda_1=1$, $\lambda_2=-1$ und $\lambda_3=2$. Aus $(\mathbf{M}_1-\lambda_i\mathds{1})\cdot\mathbf{x}_i=0$ berechnen wir die Eigenvektoren.  Für $\lambda_1=1$ erhalten wir das LGS \[
 \begin{pmatrix}-10&2&6\\5&-1&-3\\-16&4&10\end{pmatrix}\cdot
 \begin{pmatrix}x_{11}\\x_{12}\\x_{12}\end{pmatrix}=
 \begin{pmatrix}0\\0\\0\end{pmatrix}
\]
Dies ist ein LGS, dessen Rang der Koeffizientenmatrix 2 ist, gleich dem Rang der erweiterten Koeffizientenmatrix. Damit erwarten wir eine parametrische Lösung mit einem frei wählbaren Parameter.
Wir sehen, dass die ersten beiden Zeilen der Koeffizientenmatrix linear abhängig sind (multipliziere 2. Zeile mit $-2$ und erhalte Zeile 1). Davon ist allerdings die dritte Zeile unabhängig. Also hat die Koeffizientenmatrix den Rang 2. Der Rangabfall gegenüber $\mathbf{M}_1$ ist 1 und dies ist gleich der Vielfachheit des Eigenwertes, im voeliegenden Fall 1. In der Tat haben wir 3 verscheidene Eigenwerte.

Die Eigenvektoren $\mathbf{x}_i$ lauten daher
\[
\mathbf{x}_1=\begin{pmatrix}1\\ -1\\ 2\end{pmatrix},\quad
\mathbf{x}_2=\begin{pmatrix}2\\ -1\\ 3\end{pmatrix}\text{ und }
\mathbf{x}_3=\begin{pmatrix}2\\ -1\\ 4\end{pmatrix}.
\]
Wir haben die Komponenten so gewählt, dass sie ganze Zahlen ergeben, das ist natürlich will\-kürlich bei einer Parameterlösung.

Die Eigenvektoren bilden die Transformationsmatrix $\mathbf{P}_1$, mit deren Hilfe wir $\mathbf{M}_1$ auf Diagonalgestalt bringen und zwar mit $\bm{\Lambda}_1=
\mathbf{P}_1^{-1}\mathbf{M}_1\mathbf{P}_1$. Es ist
\[
\mathbf{P}_1=\begin{pmatrix}1&2&2\\ -1&-1&-1\\ 2&3&4\end{pmatrix}
\quad\text{ und }\quad
\mathbf{P}_1^{-1}=\begin{pmatrix}-1&-2&0\\ 2&0&-1\\ -1&2&1\end{pmatrix}.
\]
Wir erhalten somit
\[
\bm{\Lambda}_1=
\begin{pmatrix}-1&-2&0\\ 2&0&-1\\ -1&2&1\end{pmatrix}
\begin{pmatrix}-9&2&6\\5&0&-3\\-16&4&11\end{pmatrix}
\begin{pmatrix}1&2&2\\ -1&-1&-1\\ 2&3&4\end{pmatrix}=
\begin{pmatrix}1&0&0\\0&-1&0\\0&0&2\end{pmatrix}
\]

Die Eigenwerte von $\mathbf{M}_2$ sind $\lambda_1=7$ und $\lambda_2=\lambda_3=-2$. wir haben einen doppelt auftretenden Eigenwert. Für $\lambda_1=7$ erhalten wir den Eigenvektor $\mathbf{x}_1=(2/3,2/3,-1/3)^T$. Was passiert nun mit dem doppelten Eigenwert, können wir hierfür Eigenvektoren identifizieren? Wir untersuchen die Matrix  $\mathbf{M}_2-\lambda_{2}\mathds{1}$ und stellen fest dass diese einen Rangabfall von 2 aufweist. Damit erhalten wir für die Lösung des Gleichungssystems $(\mathbf{M}_2-\lambda_{2}\mathds{1})\cdot\mathbf{x}_i=0$ eine zweiparametrische Lösung und wir finden die zwei linear unabhängigen Eigenvektoren
\[
\mathbf{x}_2=\frac{\sqrt{2}}{2}
\begin{pmatrix}1\\ -1\\ 0\end{pmatrix}
\quad\text{ und }
\mathbf{x}_3=\frac{\sqrt{2}}{6}
\begin{pmatrix}1\\ 1\\ 4\end{pmatrix}
\]
Die Konstruktion von $\mathbf{P}_2$ und $\mathbf{P}_2^{-1}$ mit diesen Eigenvektoren erlaubt uns die Transformation von $\mathbf{x}_2$ auf Diagonalform
\[
\bm{\Lambda}_2=
\begin{pmatrix}7&0&0\\0&-2&0\\0&0&-2\end{pmatrix}.
\]

Eine ganz andere Situation liegt bei $\mathbf{M}_3$ vor. Wir berechnen die Eigenwerte und stellen fest, dass $\lambda_1=\lambda_2=\lambda_3=-1$.
Die Eigenvektoren berechnen wir aus  $(\mathbf{M}_3+\mathds{1})\cdot\mathbf{x}_i=0$. Wir stellen fest, dass $\text{Rg}(\mathbf{M}_3+\mathds{1})=2$ nicht einen Rangabfall von 3 aufweist, wie es bei einem dreifachen Eigenwert der Fall sein müsste. Also haben wir nur eine zweiparametrische Lösung vorliegen, die wir aus einer der linear abhängigen Gleichungen $(\mathbf{M}_3+\mathds{1})\cdot\mathbf{x}_i=0$ berechnen können:
\[
\begin{pmatrix}7&-6&5\\ 14&-12&10\\ 7&-6&5\end{pmatrix}
\begin{pmatrix}x_{i1}\\ x_{i2}\\ x_{i3}\end{pmatrix}=
\begin{pmatrix}0\\ 0\\ 0\end{pmatrix}
\]
Wir sehen sofort, dass diese drei Gleichungen linear abhängig sind 
\[ 7x_{i1}-6x_{i2}+5x_{i3}=0\]
und wir nur eine zweiparametrische Lösung bekommen. Wir wählen schlauerweise $x_{i1}=5\cdot s_i$ und $x_{i2}=5\cdot t_i$ und erhalten damit $x_{i3}=-7\cdot s_i+6\cdot t_i$. Für $s_1=\frac{1}{5}$ und $t_1=\frac{2}{5}$ ergibt sich der Eigenvektor $\mathbf{x}_1=(1,2,1)^T$. Mit $s_2=\frac{3}{5}$ und $t_2=\frac{1}{5}$ ergibt sich der Eigenvektor $\mathbf{x}_2=(3,1,-3)^T$. Alle weiteren Eigenvektoren sind Linearkombination aus $\mathbf{x}_1$ und $\mathbf{x}_2$. D.h.\  das LGS gibt keine weiteren Eigenvektoren her. $\mathbf{P}_3$ wäre singulär und nicht invertierbar. Daher können wir $\mathbf{M}_3$ nicht auf Diagonalform bringen. Berechnen wir anstelle eines linear abhängigen dritten Eigenvektors den Lösungsvektor von $(\mathbf{M}_3+\mathds{1})^2\cdot\mathbf{x}_i=0$, dann lässt sich zeigen, dass ein $\mathbf{P}_3$ gebildet aus den ersten beiden Eigenvektoren und diesem Lösungsvektor $\mathbf{M}_3$ zumindest auf die \textsc{jordan}sche Normalform transformieren kann. Dies ist keine Diagonalform. Auf der Diagonalen stehen zwar die Eigenwerte, aber die obere Nebendiagonale ist mit Einsen besetzt. Bei $k$-fachem Eigenwert und Rangabfall $r\le k$ stehen genau $k-r$ Einsen auf der oberen Nebendiagonale. Diese Aussage beweisen wir nicht. Berechnen wir zunächst den Lösungsvektor von $(\mathbf{M}_3+\mathds{1})^2\cdot\mathbf{x}_i=0$. Wir sehen, dass $(\mathbf{M}_3+\mathds{1})^2$ die Nullmatrix ist und damit die Lösung des Gleichungssystems beliebig. Ganz beliebig geht natürlich nicht, der Vektor muss zumindest linear unabhängig von den ersten beiden Eigenvektoren sein. Wir wählen $\mathbf{x}_3=(0,1,1)^T$ und konstruiren uns die Transformationsmatrix
\[
\mathbf{P}_3=
\begin{pmatrix}-1&0&3\\ -2&1&1\\ -1&1&-3\end{pmatrix}\quad\text{ wobei }\quad
\mathbf{P}_3^{-1}=
\begin{pmatrix}-4&3&-3\\ -7&6&-5\\ -1&1&-1\end{pmatrix}.
\]
Damit erhalten wir 
\[
\mathbf{P}_3^{-1}\mathbf{M}_3\mathbf{P}_3=
\begin{pmatrix}-1&1&0\\ 0&-1&0\\ 0&0&-1\end{pmatrix}.
\]
\newpage
\section{Das Computeralgebrasystem Maxima}
Maxima ist ein Computeralgebrasystem (CAS). Der Quellcode wurde seit 1982 von William Shelter\footnote{William Frederick Schelter (1947 – July 30, 2001) war Professor für Mathematik der  University of Texas in Austin. Er war Lisp Entwickler und  Programmierer (siehe \url{https://en.wikipedia.org/wiki/Bill_Schelter}).}  entwickelt und 1998 unter der GNU General Public License (GPL) veröffentlicht.

Maxima ist die Engine, die hinter den Stackaufgaben auf ILIAS steht, die wir im Rahmen dieses Kurses bearbeiten. Es ist daher ratsam, sich zumindest mit der Syntax von Maxima zu beschäftigen, da diese zur Eingabe bei den Aufgaben benutzt werden muss. Die Syntax ist einigermaßen intuitiv, es schadet aber nicht, sich einmal damit zu beschäftigen. Weitere Information ist hier 
\url{http://maxima.sourceforge.net/docs/manual/de/maxima.html#SEC_Top} 
zu finden.
\newpage