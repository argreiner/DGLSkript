\chapter{Nichtlineare gewöhnliche Differentialgleichungen}
Dieser Abschnitt behandelt Beispiele nichtlinearer gewöhnlicher
Differentialgleichungen und deren Lösung mit verschiedenen Zugängen. Zuerst
sollen nichtlineare Systeme erster Ordnung und ersten Grades behandelt werden,
die eine eindeutige Zeitskalentrennung der abhängigen veränderlichen aufweisen
und bei denen es möglich ist, das qualitative Lösungsverhalten durch wenige
instabile Freiheitsgrade auszudrücken ({\red H.Haken, Advanced Synergetics,
Kapitel 7}). Dies führt auf die sogenannten Ordnungsparametergleichungen.
Danach wird auf die numerische Lösung von Differentialgleichungen eingegangen.
\section{Einleitung}
Die Differentialgleichung $\dot q=\alpha q$ ist linear und von erster Ordnung,
während die Gleichung $\ddot q+\omega^2 q=0$ eine lineare Differentialgleichung
2. Ordnung ist, die wir auf ein System von zwei Differentialgleichungen erster
Ordnung und ersten Grades zurückführen können. Wir schreiben
\[
  \begin{matrix}\dot q_1&=q_2\\ \dot q_2&= -\omega^2 q_1 \end{matrix} 
  \quad\text{ oder allgemein }\quad\dot{\mathbf{q}}=L\cdot\mathbf{q}
\]
$L$ bezeichnet hierbei die Koeffizientenmatrix, wir haben es mit einem linearen
System von Diffeentialgleichungen zu tun.

Jetzt erweitern wir das Konzept auf nichtlineare Differentialgleichungen,
genauer auf ein System nichtliearer Differentialgleichungen erster Ordnung und
ersten Grades
\begin{align*}
\dot q_1 =& \alpha q_1 + \beta q_1q_2\\
\dot q_1 =&\dots
\end{align*}
Hierbei ist $\beta$ ein Kontrollparameter, der die nichtlineare Kopplung
zwischen den Freiheitsgraden $q_1$ und $q_2$ kontrolliert. Allgemein schreiben
wir so etwas als
\begin{equation}
  \quad\dot{\mathbf{q}}=N(\mathbf{q})
  \label{eq:NLDGLSystem}
\end{equation}
\begin{example}{Nichtlineare Differentialgleichungen}
  \begin{enumerate}
   \item Der van der Pol Oszillator
    \[\ddot x(t)+\mu(x(t)^2-1)\dot x(t)+\omega^2x(t)=0 \]
    den wir wiederum umschreiben zu 1. Ordnung
    \begin{align*}
     \dot y_1=&y_2\\
     \dot y_2=&\mu(1-y_1(t)^2) y_2(t)-\omega^2y_1(t)
    \end{align*}
    Wie sieht das Phasenraumbild aus?
   \item Der Lorenz Attraktor
     \begin{align*}
       \dot x=&\sigma(y-x)\\
       \dot y=&x(\rho-z)-y\\
     \dot z=&xy-\beta z
    \end{align*}
     In Wikipedia steht zu lesen:
     ``Formuliert wurde das System um 1963 von dem Meteorologen Edward N. Lorenz
     (1917–2008), der es als Idealisierung eines hydrodynamischen Systems
     entwickelte. Basierend auf einer Arbeit von Barry Saltzman (1931–2001)
     ging es Lorenz dabei um eine Modellierung der Zustände in der
     Erdatmosphäre zum Zweck einer Langzeitvorhersage. Allerdings betonte
     Lorenz, dass das von ihm entwickelte System allenfalls für sehr begrenzte
     Parameterbereiche von $\sigma$, $\rho$ und $\beta$ realistische
      Resultate liefert.'' (siehe https://de.wikipedia.org/wiki/Lorenz-Attraktor).
   \item Das Volterra-Lotka System der Populationsdynamik
     \begin{align*}
       \dot x=&x-xy\\
       \dot y=&-y+xy
     \end{align*}
  \end{enumerate}
\end{example}
Finden Sie ein Darstellung der Phasenraumbilder der drei Systeme aus obigen
Beispiel!
\section{Ordnungsparametergleichungen}
Wir betrachten folgendes System gewöhnlicher Differentialgleichungen erster
Ordnung und ersten grades in den Variablen $u(t)$ und $s(t)$.
\begin{align} 
  \dot{u}&=\alpha u-us
  \label{eq:unstable}\\
  \dot{s}&=-\beta s+u^2
  \label{eq:stable}
\end{align}
Wir nehmen an $\alpha\ge0$ und $\beta>0$. Wir nennen $u$ die instabile und $s$
die stabile Mode - das erklären wir später genauer. Erstmal wollen wir die $s$
durch die $u$ ausdrücken, indem wir so tun, als wären letztere bekannt. Dann
erhalten wir - Variation der Konstanten anwenden - folgende Lösung
\begin{equation}
  s(t)=\int\limits_{-\infty}^{t}e^{-\beta(t-\tau)}u(\tau)^2d\tau
  \label{eq:stabelsolu}
\end{equation}
mit der Anfangsbedingung $\lim\limits_{t\rightarrow -\infty}s(t)=0$.
\subsection{Adiabatische Näherung}
Wenn wir (\ref{eq:stabelsolu}) partiell integrieren, erhalten wir
\begin{equation}
  s(t)=\frac{1}{\beta}u^2(t)-\frac{1}{\beta}\int\limits_{-\infty}^{t}e^{-\beta(t-\tau)}
  2(u(\tau)\dot u(\tau))d\tau
  \label{eq:sofueliminated}
\end{equation}
Nehmen wir nun an, dass $u(t)$ sich wenig ändere, so daß $\dot u$ als sehr
klein angenommen werden kann, dann liegt es nahe das Integral in
(\ref{eq:sofueliminated}) zu vernachlässigen. Damit erhalten wir

\begin{equation}
  s(t)=\frac{1}{\beta}u^2(t),
  \label{eq:sapprox}
\end{equation}
was wir auh sofort aus (\ref{eq:stable}) erhalten würden, wenn wir $\dot s=0$
setzten.

Unter welchen Bedingungen sich das  Integral in (\ref{eq:sofueliminated})
vernachlässigen lässt, müssen wir hier noch genauer untersuchen. Wir betrachten
das maximum des Ausdrucks $Max(u(\tau)\dot u(\tau))=(|u||\dot u|)_{max}$ in
(\ref{eq:sofueliminated}) und schreiben dieses vor das Integral anstatt
$(u(\tau)\dot u(\tau))$ und integrieren. Dann ist unsere Näherung sicherlich
gut, wenn gilt
\begin{equation}
\frac{(|u||\dot u|)_{max}}{\beta^2}<<\frac{|u|^2}{\beta}
  \label{eq:BedElim}
\end{equation}
oder $ |\dot u|_{max}<<\beta|u|$. Dies bedeutet, dass $u$ sich langsam ändert
im Vergleich zur durch die Diffusionskonstante $\beta$ vorgeschriebenen
Änderung. Dies ist die Interpretation der adiabatischen Näherung.
\subsection{Exakte Eliminationsprozedur}
Um die wesentlichen Merkmale diese Prozedur heraus zu arbeiten, wählen wir das
Beispiel $\alpha=0$, so dass (\ref{eq:stable}) ersetzt wird durch $\dot u=-us$.
Wir nutzen jetzt die immer noch exakte Gleichung (\ref{eq:sofueliminated}) und
substituieren darin $\dot u$ mit $-us$ und erhalten
\begin{equation}
  s(t)=\frac{1}{\beta}u^2(t)+\frac{2}{\beta}\int\limits_{-\infty}^{t}e^{-\beta(t-\tau)}
  (u(\tau)^2 s(\tau))d\tau
  \label{eq:soluelimexact}
\end{equation}
(\ref{eq:soluelimexact}) ist eine Integralgleichung für $s(t)$, die wir durch
Iteration lösen. In niedrigster Ordnung drücken wir $s(t)$ durch die Näherung
in (\ref{eq:sapprox}) aus und erhalten
\begin{equation}
  s(t)=\frac{1}{\beta}u^2(t)+\frac{2}{\beta}\int\limits_{-\infty}^{t}e^{-\beta(t-\tau)}
  \frac{1}{\beta}u(\tau)^4d\tau .
  \label{eq:soluelim1st}
\end{equation}
Um den nächsten Iterationsschritt zu erhalten, integrieren wir
(\ref{eq:soluelim1st}) partiell und erhalten 
\begin{equation}
  s(t)=\frac{1}{\beta}u^2(t)+\frac{2}{\beta^3}u(t)^4-
  \frac{8}{\beta^3}\int\limits_{-\infty}^{t}e^{-\beta(t-\tau)}
  (u^3(\tau)\dot u(\tau))d\tau .
  \label{eq:soluelim2nd}
\end{equation}
Unter dem Integral ersetzen wir wieder $\dot u$ wie oben und ersetzen $s$ durch
den zweiten Iterationsschritt (\ref{eq:soluelim2nd}) in dem wir das Integral
vernachlässigt haben und erhalten
\begin{equation}
  s(t)=\frac{1}{\beta}u^2(t)+\frac{2}{\beta^3}u(t)^4-
  \underbrace{
    \frac{8}{\beta^3}\int\limits_{-\infty}^{t}e^{-\beta(t-\tau)}u^4(\tau)
  \left(\frac{1}{\beta}u^2(\tau)+\frac{2}{\beta^3}u(\tau)^4\right)d\tau .
  }_{I}
  \label{eq:soluelim2ndrep}
\end{equation}
Wir behalten vom Integral $I$ aus (\ref{eq:soluelim2ndrep}) zunächst nur die
Terme 6-ter Ordnung
\[ 
  I=\frac{8}{\beta^4}\int\limits_{-\infty}^{t}e^{-\beta(t-\tau)}
  u^6(\tau)d\tau+\text{ h.O.}
\]
Eine erneute partielle Inegration führt auf
\[ 
  s(t)=\frac{1}{\beta}u^2(t)+\frac{2}{\beta^3}u(t)^4+\frac{8}{\beta^5}u(t)^6+\dots
\]
Es ist offensichtlich, dass wir diese Iteration immer weiter betreiben und
somit $s(t)$ als eine Potenzreihe in $u$ ausdrücken können. Wenn $u$ klein
genug ist, können wir erwarten $s$ durch wenige Terme in $u$ annähern zu
können.

Dieses Verfahren können wir auch für allgemeinere Gleichungen als $\dot u=-us$,
wie in unserem Beispiel verwendet, anwenden. Für diese Diskussion sei auf die
Literatur verwiesen. Was wir hier festhalten wollen, ist die Tatsache, dass
unter der oben genannten Bedingung (\ref{eq:BedElim}), eine adiabatische
Elimination der stabilen Moden eines Systems durchgeführt werden kann. 

Noch einmal: warum nennen wir $s$ eine stabile Mode? $\beta$ ist sehr groß und
daher relaxiert $s$ sehr schnell auf den stationären Wert, der nun aber durch
$u$ vorgegeben wird. Das ist eine sehr vage Behauptung und diese muss daher
durch eine genaue Untersuchung der Konvergenz der Potenzreihenentwicklung
geklärt werden.
\section{Numerische Verfahren}
Nichtlineare Differentialgleichungen sind im Allgemeinen nicht analytisch
lösbar. D.h.\ es ist nicht möglich eine geschlossene Lösung anzugeben. Daher
ist es sinnvoll, sich mit numerischen Lösungsmethoden zu beschäftigen.  Wir
wollen hier nur eine begrenzte auswahl ansprechen. Für ingneieurstechnische
Anwendungen steht ganz klar das Verhältnis zwischen Aufwand für deren
Erarbeitung und Genauigkeit der Ergebnisse im Vordergrund. Deshalb gibt es
keine absolute Aussage, welche Methode die beste ist.

Wir gehen von einem Anfangswertproblem einer Differentialgleichung 1. Ordnung
und 1. Grades aus, wobei hierin auch System erster Ordnung eingeschlossen
seien. Wir schreiben das Anfangswertproblem eines Systems, wie es in
(\ref{eq:NLDGLSystem}) dargestellt ist als
\begin{align}
  \dot{\mathbf{y}}(t) =& \mathbf{f}(\mathbf{y}(t),t) \label{eq:yNLSystem}\\
  \mathbf{y}(0)       =& \mathbf{y}_{0}\nonumber
\end{align}
\subsection{Die Eulersche Methode}
Wir nehmen an $y(t)$ sei gekannt, dann können wir $y(t+\Delta t)$ in eine
Potenzreihe entwickeln und erhalten bei Vernachlässigung aller Terme 2. und
höherer Potenzen in $\Delta t$ 
\begin{equation}
  y(t+\Delta t) = y(t)+\Delta t f(y(t),t)+O(\Delta t^2),
  \label{eq:Euler}
\end{equation}
wobei $O(\Delta t^2)$ für quadratische und höhere Terme steht.

\begin{note}{Fehler}
  Keine Angabe eines Algorithmus zur Lösung einer DGL ist sinnvoll, wenn nicht
  gleichzeitig auch zumindest die Fehlerordnung, wenn nicht gar der lokale
  Dikretisisierungsfehler angegeben ist.
\end{note}
Wie groß ist der Fehler, den wir bei der Lösung machen, wenn wir
(\ref{eq:Euler}) für gleiche Zeitabstände zu diskreten Zeitpunkten
$t_i$ anwenden? Zumindest in einem Zeitschritt gelingt uns sofort eine Aussage.
Angenommen $Y_i$ sei die exakte Lösung der Differentialgleichung zum Zeitpunkt
$t_i$. Dann können wir $Y_{i+1}$ ebenfalls durch eine Taylorreihe darstellen
\begin{equation}
  Y_{i+1}=Y_{i}
  +\underbrace{\Delta t \left(\frac{dY}{dt}\right)_i}_{f(Y_i,t_i)}
  +\underbrace{\frac{\Delta t^2}{2}\left(\frac{d^2Y}{dt^2}\right)_i}_{\dot{f}(Y_i,t_i)}+\dots
  \label{eq:TaylorExact}
\end{equation}
Wobei wir jetzt auf der rechten Seite von (\ref{eq:TaylorExact}) $Y_i$ durch
$y_i$ ersetzen und von (\ref{eq:Euler}) die Potenzreihe (\ref{eq:TaylorExact})
abziehen. Damit erhalten wir für den Fehler
\begin{equation}
  y_{i+1}-Y_{i+1}=E_{i+1}=-\frac{\Delta t^2}{2}\dot{f}(Y_i,t_i)O(\Delta t^3)
  \label{eq:LokalerFehler}
\end{equation}
Dies ist der Fehler pro Zeitschritt, auch lokaler Diskretisierungfehler genannt.
\subsection{Runge Kutta Methode}
Wir gehen von einer zentralen finiten Differenzenformel aus. Das bedeutet, dass
die finite Differenz die Ableitung zum Zeitpunkt $t_i+\frac{\Delta t}{2}$
nähert. Wir erhalten
\begin{equation}
  \frac{y_{i+1}- y_i}{\Delta}=\dot{y}_{i+1/2}= f(y_{i+1/2},t_{i+1/2})
  \label{eq:Midpoint}
\end{equation}
wobei aber $y_{i+1/2}$ unbekannt ist. Dies beschaffen wir uns durch
Vorwärtsintegration um einen halben Zeitschritt $\frac{\Delta t}{2}$, somit
\begin{align*}
  \hat{y}_{i+1/2}=&y_i+\frac{\Delta t}{2}f(y_{i},t_{i})\\
  t_{i+1/2} =& t_i+\frac{\Delta t}{2}\\
  y_{i+1} =& y_{i}+\Delta t f(\hat{y}_{i+1/2},t_{i+1/2})
\end{align*}
Dies ist das sogenannte Runge-Kutta-Schema 2. Ordnung. Wir sehen, dass wir
hier, im Gegensatz zum Euler-Methode die Funktion $f$ zweimal auswerten müssen.

Ein verbessertes Schema ist die Runge-Kutta-Methode 4. Ordnung. Wie der name
sagt, müssen wir hier die  Funktion $f$ viermal auswerten. Wir erhalten
\begin{align}\label{eq:RK4order}
  y_{i+1} =& y_{i}+\frac{\Delta t}{6}\left( 
    f(y_i,t_i)+2f(\hat{y}_{i+1/2},t_{i+1/2})+2f(\hat{\hat{y}}_{i+1/2},t_{i+1/2})
  +f(\hat{y}_{i+1},t_{i+1})\right)\\
  \hat{y}_{i+1/2}=&y_i+\frac{\Delta t}{2}f(y_{i},t_{i})\nonumber\\
  \hat{\hat{y}}_{i+1/2}=&y_i+\frac{\Delta t}{2}f(y_{i+1/2},t_{i+1/2})\nonumber\\
  \hat{y}_{i+1}=&y_i+\Delta tf(\hat{\hat{y}}_{i+1/2},t_{i+1/2})\nonumber
\end{align}

\subsection{Mehrschrittmethoden}
Die Idee hinter den Mehrschrittmethoden ist, (\ref{eq:NLDGLSystem}) zwischen
$t$ und $t+\Delta t$ zu integrieren
\begin{equation*}
y(t_{n+s})=y(t_{n+s-1})+\int\limits_{t_{n+s}}^{t_{n+s-1}}f(y(t'),t')dt'
\end{equation*}
Das Integral gilt es nun zu nähern. Wir haben ja bereits Werte für $y(t_i)$ zu
den Zeitpunkten $t_n$  bis $t_{n+s-1}$ berechnet. Wir nehmen  diese um eine
Interpolationsfunktion aufzustellen, die wir nun in den Grenzen von $t_{n+s-1}$
bis $t_{n+s}$ integrieren können. Wir nähern $f(y(t),t)$ z.B. durch ein
Polynom $p(t)$ der Ordnung $s-1$, das wir dergestalt konstruiert haben, dass es an
den Interpolationspunkten mit $f(y(t),t)$ übereinstimmt

Das Interpolationspolynom ist analytisch integrierbar.  Explizit schreiben wir
das folgendermaßen
\[ p(t)=\sum\limits_{m=0}^{s-1}p_m(t)f(y(t_{n+m},t_{n+m}). \]
Hierbei bezeichne $p_m(t)$ ein Polynom, das zum Zeitpunkt $t_{n+m}$ gleich 1
ist und zu allen anderen Zeitpunkten 0. Damit nutzen wir $s$ vorhergehende
Zeitschritte aus. Daher auch der Name Mehrschrittverfahren und hier im
speziellen beschrieben ist die Methode von Adams.
%\subsection{Predictor-Corrector Methode}
%\subsection{Shooting- und Matrixmethode für Randwertprobleme}
