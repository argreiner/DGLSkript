\chapter[Erinnerung an die \dots]{Erinnerung an die Integral- und
Differentialrechnung und die Lineare Algebra}
In diesem Kapitel wollen wir an die in den Grundvorlesungen der Mathematik für
Ingenieurinnen und Informatiker behandelten Themen erinnern.
\section{Sätze der Differential- und Integralrechnung}
\begin{satz}{Hauptsatz der Differential- und Integralrechnung\label{theo:HSIntegralDiff}}
Wenn für eine stetige Funktion $f(x)$ das Integral
$F(x)=\int\limits_{x_0}^{x}f(\xi)d\xi$ existiert, dann ist
\[
  \frac{dF(x)}{dx}=\frac{d}{dx}\int\limits_{x_0}^{x}f(\xi)d\xi=f(x)
\]
\label{theo:HDI}
\end{satz}
Satz \ref{theo:HDI} in Worten:„Die Ableitung des Integrals ist gleich der
Funktion im Integranden an der oberen Integrationsgrenze“.
\begin{example}{Ableitung eines Integrals}
\[
 \frac{d F(x)}{dx}=\frac{d}{dx}\int\limits_{x_0}^{x}e^{-\xi}d\xi=
 \frac{d}{dx} \left(-e^{-x}+e^{-x_0}\right)=e^{-x}
\]
\end{example}
\begin{satz}{Mittelwertsatz der Integralrechnung}
  Gegeben eine Funktion $f(x)$, die auf dem Interval $[a,b]$ stetig ist, sowie
  eine Funktion $g(x)$ für die im Intervall $[a,b]$ entweder $g(x)\ge0$ oder
  $g(x)\le0$ gelte. Dann gibt es mindestens eine Stelle $\xi\in[a,b]$, so dass
  mit deren entsprechendem Funktionswert 
  \begin{equation} 
  \int\limits_a^bf(x)g(x)dx=f(\xi)\int\limits_a^bg(x)dx 
  \label{eq:MittelwertInt2} 
  \end{equation}
  gilt.   
\end{satz}
Wenn wir $g(x)=1 $ setzen, dann wird dies wird gemeinhin als erster
Mittelwertsatz der Integralrechnung bezeichnet. Es gilt also:
\begin{equation} 
  \int\limits_a^bf(x)dx=f(\xi)(b-a)
  \label{eq:MittelwertInt1}
\end{equation}
%\begin{example}{Anwendung von Satz \ref{eq:MittelwertInt2}}
%?
%\end{example}
\begin{satz}{Mittelwertsatz der Differentialrechnung}
  Gegeben 2 Funktionen $f(x)$ und $g(x)$, die beide auf dem Interval $[a,b]$
  stetig und differenzierbar seien. Dann existiert ein $x_0\in(a,b)$, so dass   
  \begin{equation}
    f'(x_0)\left(g(b)-g(a)\right)= g'(x_0)\left(f(b)-f(a)\right)
  \label{eq:MittelwertDiff2}
\end{equation}
\end{satz}
Wenn wir $g'(x)=1 $ setzen, dann wird dies wird gemeinhin als erster
Mittelwertsatz der Differetialrechnung bezeichnet. Es gilt also:
\begin{equation}
  f'(x_0)=\frac{f(b)-f(a)}{b-a}
  \label{eq:MittelwertDiff1}
\end{equation}
\section{Taylorreihe}
%{\red Das könnte ein Beispiel für den Beweis eines Satzes sein}
Die Taylorreihe ist zweifelsohne eines der wichtigen Instrumente. Sie dient zur
Abschätzung des Verhaltens von Funktionen und findet in den numerischen
Methoden Anwendung, wie z.B. den Finiten Differenzen.
\begin{satz}{Der Taylorsche Satz\label{theo:Taylor}} Sei $f$ eine Funktion, die
  eine $(n+1)$-te stetige Ableitung auf einem Intervall $J$ besitze. Es seien
  $a,b\in J$. Dann ist 
\[
    f(b)=f(a)+\frac{b-a}{1!}f'(a)+\cdots +\frac{(b-a)^{n}}{(n)!}f^{(n)}(a)+R_n
\]
  mit
\[
    R_n=\int\limits_a^b\frac{(b-s)^{n}}{n!}f^{(n+1)}(s)ds
\]
Des weiteren merken wir an, dass eine Zahl $c\in[a,b]$ existiert, so dass
\[
    R_n=\frac{(b-a)^{n}}{n!}f^{(n+1)}(c)
\]
gilt.
\end{satz}
{\bf Beweis:} Wir wenden Satz 1 an und integrieren partiell
\begin{align}
  f(b)=&f(a)+\int\limits_a^bf'(s)ds=f(a)+\int\limits_a^b\frac{(b-s)^0}{0!}f'(s)ds\nonumber\\
  =&f(a)-\left.\frac{(b-s)^1}{1!}f'(s)\right|_a^b+\int\limits_a^b\frac{(b-s)^1}{1!}f''(s)ds=\dots
  \label{eq:TaylorProof1}
\end{align}
Vollständige Induktion liefert den Schluss des Beweises.
\section{Lineare Algebra}
In der Mathematik treffen wir auf verschiedene Arten von Objekten, die
untereinander addiert und mit Zahlen multipliziert werden können. Eine
Ansammlung von Objekten nennen wir eine Menge. Ein Beispiel ist die Menge von
Vektoren ein und derselben Dimension. Wir wollen für all die
Spezialfälle eine allgemeine Definition geben. 
\subsection{Der Vektorraum}
Ein {\bf Vektorraum} V ist eine Menge von Objekten, die addiert und mit Zahlen
multipliziert werden können und zwar so, dass die Summe zweier Elemente aus V
wieder ein Element aus V ergibt, das Produkt eines Elementes aus V mit einer Zahl
ebenfalls wieder in V ist und folgende Eigenschaften erfüllt sind:.
\begin{description}
  \item[{\bf VR 1.}] Gegeben $u,v,w\in V$. Es soll Assoziativität gelten
    \[ (u+v)+w = u+(v+w) \] 
  \item[{\bf VR 2.}] Es gibt ein Element 0 in V, so dass gilt
    \[ 0+u = u+0 =u \]
    für alle $u\in V$. Wir nennen dieses Element den Nullvektor.
  \item[{\bf VR 3.}] Für jedes $u\in V$ gibt es ein Element $(-1)u\in V$ für das 
  \[ u+(-1)u=0 \]
    D.h. jedes Element hat ein Inverses.
  \item[{\bf VR 4.}] Für alle $u,v\in V$ soll Kommuntativität gelten
    \[ u+v = v+u \]
  \item[{\bf VR 5.}] Für eine Zahl $c$ gilt Distributivität $c(u+v)=cu+cv$ 
  \item[{\bf VR 6.}] Für zwei Zahlen $a$, $b$ gilt Assoziativität bezüglich der Addition $(a+b)u=au+bu$
  \item[{\bf VR 7.}] Für zwei Zahlen $a$, $b$ gilt Assoziativität bezüglich der Multiplikation $(ab)u=a(bu)$
  \item[{\bf VR 8.}] Für alle $u\in V$ gilt $1\cdot u=u$
\end{description}
Als einen Unterraum $W$ von $V$ bezeichnen wir einen Raum für den gilt
\begin{enumerate}
  \item $\forall u,v\in W$ ist auch $u+v\in W$
  \item $\forall u\in W$ und $c\in\mathbb{R}$ ist auch $cu\in W$
  \item $0\in V$ ist auch Element von $W$ 
\end{enumerate}
\subsection{Lineare Unabhängigkeit, Basis}
Eine Linearkombination von Vektoren $(v_1,\dots,v_m)\in V$ schreiben wir als
$a_1v_1+\dots+a_mv_m$, wobei $a_1,\dots,a_m\in\mathbb{R}$ oder auch
$a_1,\dots,a_m\in\mathbb{C}$. Alle möglichen Linearkombinationen der $v_i$
nennen wir die lineare Hülle der $(v_1,\dots,v_m)\in V$ und schreiben dafür
$span(v_1,\dots,v_m)$. Die $m$ Vektoren spannen einen Unteraum $W\subset V$
auf.
\begin{example}{Lineare Hülle}
Es ist
\[(7,2,9)=2(2,1,3)+3(1,0,1). \]
Der Vektor $(7,2,9)$ ist also eine Linearkombination von $(2,1,3)$ und
$(1,0,1)$.  Daher sagen wir auch $(7,2,9)\in span\left( (2,1,3),(1,0,1)
\right)$ und damit nennen wir diesen Vektor auch linear abhängig.
\end{example}
Wir nennen Vektoren $(v_1,\dots,v_m)\in V$ linear unabhängig, wenn keiner der
$m$ Vektoren als Linearkombination der restlichen $m-1$ Vektoren geschrieben
werden kann. Eine Basis des Vektorraums $V$ ist gegeben durch die maximale
Anzahl linear unabhängiger Vektoren, die ganz $V$ aufspannen. Diese muss nicht
eindeutig sein.
\begin{example}{Basen}
  $(1,0,0)$, $(0,1,0)$ und $(0,0,1)$ spannen der $\mathbb{R}^3$ auf. Genauso
  tut das aber auch die Basis
  $(1/\sqrt{2},-1/\sqrt{2},0)$,$(-1/\sqrt{2},1/\sqrt{2},0)$ und $(0,0,1)$.
\end{example}
\Comment{Hier wäre es gut Beispiele für Vektorräume aufzuführen, wie sie z.B. im Buch von Sheldon Axler, Linear Algebra done right, gegeben sind.}
\subsection{Der Rang einer Matrix}
Gegeben die Matrix
\begin{equation}
  A_{mn}=
  \begin{pmatrix}
    a_{11}&a_{12}&\;\cdots\;&a_{1n}\\
    a_{21}&a_{22}&\;\cdots\;&a_{2n}\\
    & &\;\cdots\;&\\
    a_{m1}&a_{m2}&\;\cdots\;&a_{mn}\\
  \end{pmatrix}
  \label{eq:Matrixmn}
\end{equation}
Wobei $s$ das Minimum der Anzahl der Zeilen und der Spalten $m$ und $n$
bezeichne. Durch Streichen von Zeilen oder Spalten erhalten wir quadratische
$s\times s$ Untermatrizen. Gehen wir davon aus, dass $A$ nicht Nullmatrix ist,
dann finden wir sicherlich unter den $s\times s$ Untermatrizen solche, deren
Determinanten von Null verschieden sind. Die Maximale Zahl $r$ an Reihen, bzw.
Spalten, der von Null verschiedenen Determinanten, die bei dieser Operation
entstehen, nennen wir den Rang der Matrix $A$.
\begin{example}{Rang einer $5\times7$-Matrix}
  Man bestimme den Rang der Matrix
  \[
    A=\begin{pmatrix}
    7&1&0&2&-1&4&5\\
    1&1&2&3& 0&1&2\\
    0&1&-2&1&2&0&1\\
    4&-1&-8&-6&1&1&0\\
    0&1&2&1&4&0&1\\
    \end{pmatrix}
  \]
  Wir wissen aus der linearen Algebra, dass der Wert einer Detrminante
  unverändert bleibt, wenn zu einer Zeile bzw. einer Spalte ein beliebiges
  Vielfaches einer Zeile bzw. einer Spalte addiert wird.

  Dies machen wir uns zunutze und addieren Vielfache der Zeilen der obigen
  Matrix sukzessive solange, bis folgende Matrix entsteht: 
  \[
    \begin{pmatrix}
    1&0&0&0&0&0\\
    0&1&0&0&0&0\\
    0&0&1&-2&-2&1\\
    0&0&0&13&11&-3\\
    \end{pmatrix}
  \]
  Beachte: alle linear abhängigen Zeilen und Spalten können wir streichen. 
  
  Als letzte Umformung erhalten wir
  \[
     \begin{pmatrix}
    1&0&0&0\\
    0&1&0&0\\
    0&0&1&0\\
    0&0&0&1\\
    \end{pmatrix}
  \]
  woraus wir den Rang $r=4$ ablesen.
\end{example}
\newpage