\chapter{Gewöhnliche lineare homogene Differentialgleichungen}
\Comment{2. Vorlesung}
\section{Gleichungen mit konstanten Koeffizienten}
Eine lineare Differentialgleichung n-ter Ordnung ist ein Zusammenhang folgender
Art
\begin{equation}\label{eq:DGLOrdnungN}
a_n\frac{d^n y(t)}{dt^n}+a_{n-1}\frac{d^{n-1} y(t)}{dt^{n-1}}+
\cdots +a_{1}\frac{d y(t)}{dt}+a_0y(t)=f(t)
\end{equation}
Dabei ist $y(t)$ eine Funktion der unabhängigen Veränderlichen $t$.
Differentialgleichungen vom Typ (\ref{eq:DGLOrdnungN}) beschreiben zahlreiche
Phänomene in Naturwissenschaft und Technik. Zunächst einmal wollen wir die
linearen Differentialgleichungen mit konstanten Koeffizienten untersuchen. Die Funktion $f(t)$ sei nur von der unabhängigen Veränderlichen $t$ ab\-hängig. Wenn $f(t)=0$ nennen wir (\ref{eq:DGLOrdnungN}) eine homogene Differentialgleichung, wenn dem nicht so ist, heisst (\ref{eq:DGLOrdnungN}) inhomogen.
\begin{note}{Homogene Funktion vom Grad n}
  Eine Funktion $f(x)$ wird homogen vom Grad $n$ genannt, wenn gilt $f(\lambda\cdot x)=\lambda^n\cdot f(x)$. Deshalb nennt man $f(t)$ der linearen Differentialgleichung (\ref{eq:DGLOrdnungN}) auch die \textit{Inhomogenität} der Differentialgleichung.
\end{note}

Man beachte, dass die Koeffizienten der Ableitungen von $y(t)$ nicht
notwendigerweise Konstanten sein müssen, sondern bekannte Funktionen $a_k(t)$
sein können. Wichtig für die Bezeichnung als lineare Differentialgleichung ist
ganz einfach die Tatsache, dass das Superpositionsprinzip gilt. 
\begin{note}{Lineare Superposition von Lösungen}
D.h. wenn $y_a(t)$ und $y_b(t)$ je eine Lösung der homogenen linearen
Differentialgleichung sind, dann ist auch $y(t)=\alpha y_a(t) + \beta y_b(t)$
eine Lösung. Es löse $y_f(t)$ die Differentialgleichung mit der Inhomogenität
$f(t)$ und $y_g(t)$ diejenige mit der Inhomogenität $g(t)$, dann löst
$y(t)=\alpha y_f(t) + \beta y_g(t)$ die Differentialgleichung mit der
Inhomogenität $h(t)=\alpha f(t) + \beta g(t)$. Man beweise diese Aussage!
\end{note}
%
\section{Lösung der Gleichungen mit konstanten Koeffizienten}
Zur Lösung der Gleichungen vom Typ (\ref{eq:DGLOrdnungN}) haben wir
verschiedene Herangeshensweisen. Dabei gehen wir davon aus, dass es genau eine
Lösung von (\ref{eq:DGLOrdnungN}) mit $f(t)=0$, also der homogenen
Differentialgleichung, gibt, die den Bedingungen 
\begin{equation}
y(t_0)=b_0,\,\dot{y}(t_0)=b_1,\,\dots y^{(n-1)}(t_0)=b_{n-1}
  \label{eq:ICsGDGL}
\end{equation}
genügt. Man möge den Beweis dieses Satzes in den einschlägigen Lehrbüchern
nachlesen.
\section{Taylorreihe und  Potentzreihenansatz}
Zur Lösung von (\ref{eq:DGLOrdnungN}) liegt es nahe, den Ansatz
\begin{equation}
  y(t)=\sum\limits_{\nu=0}^{\infty}c_\nu t^\nu
  \label{eq:AnsatzPotenz}
\end{equation}
zu versuchen. D.h. wir vermuten die Lösung in der Form (\ref{eq:AnsatzPotenz})
finden zu können. Hierzu gibt es zwei Herangehensweisen. Einmal mit Hilfe der
Taylorreihe und zum anderen durch direktes Einsetzen von
(\ref{eq:AnsatzPotenz}). 

Wir arbeiten beides am Beispiel des Relaxators aus.
\begin{example}{Taylorreihe verglichen mit Potenzreihenansatz}
  Gegeben die homogene Differentialgleichung $\dot{y}(t)+a_0y(t)=0$ mit
  Anfangsbedingung $y(0)=b_0$ - wir setzen o.B.d.A. $t_0=0$. Wir haben
  \[ \dot{y}(t)=-a_0y(t),\, \ddot{y}(t)=-a_0\dot{y}(t),\,\dots\, 
      y^{(n)}(t)=-a_0y^{(n-1)}(t)\]
  Damit erhalten wir die Ableitungen bei $t=0$
  \[ y(0)=b_0,\, \dot{y}(0)=-a_0b_0,\, \ddot{y}(0)=a_0^2b_0,\,\dots\, 
     y^{(n)}(0)=(-1)^na_0^nb_0\]
  und können die Taylorreihe für $y(t)$ um $t=0$ schreiben als
  \begin{equation}
    y(t)=b_0\sum\limits_{\nu=0}^{\infty}(-1)^\nu\frac{a_0^\nu}{\nu!}t^\nu=b_0e^{-a_0t}
    \label{eq:SoluTaylor}
  \end{equation}
Wenn wir (\ref{eq:AnsatzPotenz}) direkt einsetzen, dann erhalten wir die Lösung
durch Koeffizientenvergleich.
  \[ \sum\limits_{\nu=0}^{\infty}\nu c_\nu t^{\nu-1}+ 
      a_0\sum\limits_{\nu=0}^{\infty}c_\nu t^\nu=
      \sum\limits_{\mu=0}^{\infty}(\mu+1)c_{\mu+1}t^\mu+ 
      a_0\sum\limits_{\nu=0}^{\infty}c_\nu t^\nu=
      \sum\limits_{\nu=0}^{\infty} \left((\nu+1)c_{\nu+1}+a_0c_\nu\right) t^\nu
  \]
  Aus der Anfangsbedingung erhalten wir eine Rekursionsformel für die $c_\nu$
  \begin{equation}
    (\nu+1)c_{\nu+1}+a_0c_\nu=0
    \label{eq:Rekursion}
  \end{equation}
  Mit $y(0)=b0$ haben wir $c_0=b_0$ und finden damit die $c_\nu$ aus
  (\ref{eq:Rekursion}) 
    \begin{equation}
      c_{\nu}=(-1)^\nu\frac{a_0^\nu b_0}{\nu!}\qquad (\nu=0,1,2,\dots).
    \label{eq:PotenzKoeffizienten}
  \end{equation}
  Mit den Koeffizienten (\ref{eq:PotenzKoeffizienten}) erhalten wir die Lösung
  \begin{equation}
    y(t)=\sum\limits_{\nu=0}^{\infty}(-1)^\nu\frac{a_0^\nu b_0}{\nu!}t^\nu=b_0e^{-a_0t}
    \label{eq:SoluPotenz}
  \end{equation}
\end{example}
%
\section{Der Lösungsansatz $y(t)=e^{\lambda t}$}
Setzen wir $y(t)=e^{\lambda t}$ in (\ref{eq:DGLOrdnungN}) ein, so erhalten wir
\begin{equation}
  \lambda^n+a_{n-1}\lambda^{n-1}+\dots +a_0=0
  \label{eq:Charakteristische}
\end{equation}
Die Gleichung (\ref{eq:Charakteristische}) hat $n$ Nullstellen und somit haben
wir auch $n$ Lösungen, die wir als Superposition zur Lösung des
Anfangswertproblems angeben können. Für unseren Relaxator bedeutet dies
\[ \lambda e^{\lambda t}+a_0e^{\lambda t}=0\]
Und daraus erhalten wir die Lösung $y(t)=ce^{-a_0 t}$. Um die Anfangsbedingung
zu erfüllen, setzen wir $c=b_0$.

\begin{example}{Der Exponentialansatz für eine Gleichung 2. Ordnung}
  Gegeben die Gleichung $\ddot{y}(t)+2\dot{y}(t)-3y(t)=0$ mit den
  Anfangsbedingungen $y(0)=2$ und $\dot{y}(0)=-2$. Die charakteristische
  Gleichung (\ref{eq:Charakteristische}) lautet für diesen Fall
  $\lambda^2+2\lambda-3=0$ mit den beiden Lösungen $\lambda_1=1$ und
  $\lambda_2=-3$. Wir schreiben die Lösung der Gleichung als Superposition
  $y(t)=c_1e^{\lambda_1 t}+c_2e^{\lambda_2 t}=c_1e^{t}+c_2e^{-3 t}$. Wenn wir
  die beiden Anfangsbedingunen erfüllen, dann sind $c_1=c_2=1$ festgelegt.
  Damit ist die Lösung $y(t)=e^t+e^{-3t}$.
\end{example}
Damit erhebt sich die berechtigte Frage, warum man den Potenzreihenansatz dann
überhaupt in Betracht zieht. Wir erörten dies an folgendem Beispiel:
\begin{example}{Mehrfachlösungen des charakteristishen Polynoms}
  \[\ddot{y}(t)-2\dot{y}(t)+y(t)=0,\mbox{ mit } y(0)=1,\quad \dot{y}(0)=2\]
Mit dem Exponentialansatz $e^{\lambda t}$ erhalten wir das charakteristische
Polynom $\lambda^2-2\lambda+1=(\lambda -1)^2=0$. Dieses hat eine
doppelte Nullstelle, aber wie lautet die zweite Lösung?
\end{example}
Versuchen wir es mit dem Potenzreihenansatz (\ref{eq:AnsatzPotenz}) wir
erhalten die Rekursionsformel
\[ (\nu+1)(\nu+2)c_{\nu+2}-2(\nu+1)c_{\nu+1}+c_\nu=0\]
Mit $d_\nu=c_{\nu}\nu!$ erhalten wir die Rekursionsformel
$d_{\nu+2}-2d_{\nu+1}+d_\nu=0$, wobei wegen der Anfangsbedingungen $d_0=c_0=1$
und $d_1=c_1=2$ sein muss und wir sehen dass $d_\nu=\nu+1$. Damit ist aber
\[ 
  y(t)=\sum\limits_{\nu=0}^{\infty}\frac{1+\nu}{\nu!}t^\nu=
        \sum\limits_{\nu=0}^{\infty}\frac{t^\nu}{\nu!}
       +\sum\limits_{\nu=1}^{\infty}\frac{t^\nu}{(\nu-1)!}=
        e^t+t\sum\limits_{\mu=0}^{\infty}\frac{t^\mu}{\mu!}=e^t+te^t
\]       
Dies legt die Vermutung nahe, dass bei einer r-fachen Lösung $\lambda_1$ des
charakteristischen Polynoms neben $e^{\lambda_1 t}$ auch $te^{\lambda_1
t},\dots,t^{r-1}e^{\lambda_1 t}$ Lösungen sind. Man zeige dies durch einsetzen
der einzelnen Lösungen in die homogene Differentialgleichung
$(D-\lambda_1)^r[y(t)]=0$!
%
\section{Das Fundamentalsystem von Lösungen}
Die Frage nach der Zahl der Lösungen soll im folgenden behandelt werden, d.h.
wir suchen das Fundamentalsystem von Lösungen zu (\ref{eq:DGLOrdnungN}).
\begin{example}{Fundamentalsystem der Gleichung 2. Ordnung}
  Wir wissen, dass für die Gleichung 2. Ordnung 
  \[\ddot{y}(t)+a_0y(t)=0 \]
  zwei Bedingungen vorliegen müssen, damit die Lösung eindeutig angegebn werden
  kann. Also nehmen wir die Bedingunen
  \begin{align}
  y(0)=&b_0\mbox{ und }\dot{y}(0)=0\mbox{ oder}\nonumber\\
  y(0)=&0\mbox{ und }\dot{y}(0)=b_1\nonumber\\
  \end{align}
  Der Potenzreihenansatz
  \[ y(t)=\sum\limits_{\nu=0}^\infty c_\nu t^\nu \]
  führt auf
  \begin{align*}
  \sum\limits_{\nu=2}^\infty\nu(\nu-1)c_\nu t^{\nu-2}+a_0\sum\limits_{\nu=0}^\infty c_\nu t^\nu &=0\\
  \sum\limits_{\nu=0}^\infty\left( (\nu+2)(\nu+1)c_{\nu+2} +a_0c_\nu\right) t^\nu &=0
  \end{align*}
  und mit der ersten Anfangsbedingung auf die Lösungen $y_1(t)$ mit
  \[ c_{2\mu}=(-1)^\mu\frac{a_0^\mu b_0}{(2\mu)!},\quad c_{2\mu+1}=0\]
  während die zweite Anfangsbedingung auf die Lösung $y_2$ mit
  \[ c_{2\mu}=0, \quad c_{2\mu+1}=(-1)^\mu\frac{a_0^\mu b_1}{(2\mu+1)!}\]
  führt. Damit erfüllt die Summe der beiden Lösungen die allgemeine
  Anfangsbedingung
  \[ y(0)=b_0\mbox{ und }\dot{y}(0)=b_1\]
\end{example}
Das obige Beispiel legt nahe, bei homogenen Differentiagleichungen n-ter
Ordnung, wie in (\ref{eq:DGLOrdnungN}) mit $f(t)=0$ angegeben, ähnlich
vorzugehen, indem wir der Reihe nach folgende Anfangsbedingungen aufstellen
\begin{align*}
  y(t_0)=1,\dot{y}(t_0)=0,&\dots\dots\dots,y^{(n-1)}(t_0)=0;\\
  y(t_0)=0,\dot{y}(t_0)=1,&\dots\dots\dots,y^{(n-1)}(t_0)=0;\\
  &\dots\dots\dots \\
  y(t_0)=0,\dot{y}(t_0)=0,&\dots\dots\dots,y^{(n-1)}(t_0)=1;
  \label{eq:NICs}
\end{align*}
Wir bezeichnen die entsprechenden Lösungen mit $y_1(t),y_2(t),\dots,y_n(t)$.
Für die Anfangsbedingung
$y(t_0)=b_0,\dot{y}(t_0)=b_1,\dots,y^{(n-1)}(t_0)=b_{n-1}$ können wir damit
sofort die Lösung in der Form
\begin{equation}
  y(t)=b_0y_1(t)+b_1y_2(t)+\dots+b_{n-1}y_n(t)
  \label{eq:FundamentalSuperpose}
\end{equation}
angeben. Dies ist allerdings bisher auf die Stelle $t_0$ beschränkt, wir wollen
versuchen eine allgemeine Aussage zu machen. Wir bilden mit den obigen Lösungen 
\[y(t)=c_1y_1(t)+c_2y_2(t)+\dots+c_ny_n(t)\qquad (c_1,c_2,\dots,c_n\in\mathbb{R})\]
was wieder Lösung der Differentialgleichung (\ref{eq:DGLOrdnungN}) ist.

Das lineare Gleichungssystem
\begin{align}
c_1y_1(t_0)+c_2y_2(t_0)&+{\dots\dots\dots}+c_ny_n(t_0)=b_0\nonumber\\
c_1\dot{y}_1(t_0)+c_2\dot{y}_2(t_0)&+{\dots\dots\dots}+c_n\dot{y}_n(t_0)=b_1\nonumber\\
c_1\ddot{y}_1(t_0)+c_2\ddot{y}_2(t_0)&+{\dots\dots\dots}+c_n\ddot{y}_n(t_0)=b_2\nonumber\\
  &\dots\dots\dots \\
  c_1y^{(n-2)}_1(t_0)+c_2y^{(n-2)}_2(t_0)&+{\dots\dots\dots}+c_ny^{(n-2)}_n(t_0)=b_{n-2}\nonumber\\
  c_1y^{(n-1)}_1(t_0)+c_2y^{(n-1)}_2(t_0)&+{\dots\dots\dots}+c_ny^{(n-1)}_n(t_0)=b_{n-1}\nonumber
  \label{eq:LGSatt0}
\end{align}
hat die eindeutige Lösung $c_1=b_0$, $c_2=b_1$,\dots, $c_n=b_{n-1}$. Das
bedeutet dass die Koeffizientendeterminante
\begin{equation}
  W(t_0)=\begin{vmatrix}
           y_1(t_0)&y_2(t_0)&\quad\dots\quad &y_n(t_0)\\
           \dot{y}_1(t_0)&\dot{y}_2(t_0)&\quad\dots\quad &\dot{y}_n(t_0)\\
           \ddot{y}_1(t_0)&\ddot{y}_2(t_0)&\quad\ddots\quad &\ddot{y}_n(t_0)\\
	   \dots&\dots &\quad\dots\quad&\dots\\
           y^{(n-2)}_1(t_0)&y^{(n-2)}_2(t_0)&\quad\dots\quad &y^{(n-2)}_n(t_0)\\
           y^{(n-1)}_1(t_0)&y^{(n-1)}_2(t_0)&\quad\dots\quad &y^{(n-1)}_n(t_0)
         \end{vmatrix}
  \label{eq:Wronskiant0}
\end{equation}
von null verschieden ist. Die Determinante (\ref{eq:Wronskiant0}), wird
Wronskideterminante genannt und sie ist also, wenn eine Lösung für die
Anfangsbedingungen bei $t_0$ vorliegt, von null verschieden. Wenn dies aber für
beliebige Stellen $t$ der Fall sein soll, müssen wir das allgemeine Verhalten
der Wronskideterminante untersuchen, d.h. sie muss für jedes beliebige $t$ von
null verschieden sein.

Dazu bilden wir die Ableitung von $W(t)$ und erhalten
\begin{align}
  \dot{W}(t)=&\begin{vmatrix}
    \dot{y}_1(t)&\quad\dots\quad &\dot{y}_n(t)\\
           \dot{y}_1(t)&\quad\dots\quad &\dot{y}_n(t)\\
           \ddot{y}_1(t)&\quad\ddots\quad &\ddot{y}_n(t)\\
                       &\quad\dots\quad&\dots\\
           y^{(n-1)}_1(t)&\quad\dots\quad &y^{(n-1)}_n(t)
         \end{vmatrix}+
	 \begin{vmatrix}
           y_1(t)&\quad\dots\quad &y_n(t)\\
           \ddot{y}_1(t)&\quad\dots\quad &\ddot{y}_n(t)\\
           \ddot{y}_1(t)&\quad\dots\quad &\ddot{y}_n(t)\\
                       &\quad\dots\quad&\dots\\
           y^{(n-1)}_1(t)&\quad\dots\quad &y^{(n-1)}_n(t)
         \end{vmatrix}+\dots\\
      \dots+&\begin{vmatrix}
           y_1(t)&\quad\dots\quad &y_n(t)\\
           \dot{y}_1(t)&\quad\dots\quad &\dot{y}_n(t)\\
                       &\quad\dots\quad&\dots\\
           y^{(n-1)}_1(t)&\quad\dots\quad &y^{(n-1)}_n(t)\\
           y^{(n-1)}_1(t)&\quad\dots\quad &y^{(n-1)}_n(t)
         \end{vmatrix}+
      \begin{vmatrix}
           y_1(t)&\quad\dots\quad &y_n(t)\\
           \dot{y}_1(t)&\quad\dots\quad &\dot{y}_n(t)\\
                       &\quad\dots\quad&\dots\\
           y^{(n-2)}_1(t)&\quad\dots\quad &y^{(n-2)}_n(t)\\
           y^{(n)}_1(t)&\quad\dots\quad &y^{(n)}_n(t)
         \end{vmatrix}
  \label{eq:WronskianDerivative1}
\end{align}
Nur die letzte der resultierenden Determinanten ist von null verschieden.
O.b.d.A. nehmen wir $a_n=1$ in (\ref{eq:DGLOrdnungN}) an\footnote[1]{Wir teilen
(\ref{eq:DGLOrdnungN}) durch $a_n$, vorausgesetzt $a_n\ne0$, denn wenn das
nicht der Fall ist, liegt eine Differentialgleichung (n-1)-ter Ordnung vor.}
und erhalten damit
\begin{equation}
  \begin{vmatrix}
           y_1(t)&\quad\dots\quad &y_n(t)\\
           \dot{y}_1(t)&\quad\dots\quad &\dot{y}_n(t)\\
                       &\quad\dots\quad&\dots\\
      -a_{n-1}y^{(n-1)}_1(t)-\dots-a_0y_1&\quad\dots\quad 
      &-a_{n-1}y^{(n-1)}_n(t)-\dots-a_0y_n(t)
         \end{vmatrix}
  \label{eq:WronskianDerivative2}
\end{equation}
Nun multiplizieren wir die erste Zeile von (\ref{eq:WronskianDerivative2}) mit
$a_0$ und addieren sie zur letzten Zeile, sukzessive bis $a_{n-2}$
multipliziert mit den vorletzten Zeile und wir erhalten
\begin{equation}
  \dot{W}(t)=-a_{n-1}W(t)
  \label{eq:WronskianDGL}
\end{equation}
Für diese Differentialgleichung haben wir eine Anfangsbedinung $W(t_0)=w_0$ und
können sie damit lösen. Wir erhalten
\begin{equation}
  W(t)=w_0e^{-a_{n-1}(t-t_0)}
  \label{eq:WronskianGeneral}
\end{equation}
Da die Exponentialfunktion keine Nullstellen hat, schließen wir aus
(\ref{eq:WronskianGeneral}), dass wenn $W(t)$ an irgendeiner Stelle von null
verschieden ist, sie üeberall von null verschieden ist. Wir sehen also, da
$W(t)$ im gegebenen Fall nicht null werden kann, dass immer eine Lösung
existiert.
\begin{note}{}
  Beachte das über multiple Lösungen des charakteristischen Polynoms oben
  gesagte!
\end{note}