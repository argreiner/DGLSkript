\chapter{Partielle Differentialgleichungen}
Partielle Differentialgleichungen sind Differentialgleichungen mit mehr als
einer unabhängigen Variablen.  Als Beispiel stellen wir uns ein zeitabhängiges
Wärmetransportproblem in einer Dimension vor. Dieses wird mit einer
Diffusionsgleichung für die lokale Temperatur des Systems dargestellt. Die
Temperatur wird daher als Funktion zweier unabhängiger Variablen, der Zeit $t$
und der räumlichen Position $x$ dargestellt
\[\frac{\partial T(x,t)}{\partial t}=\kappa\frac{\partial^2 T(x,t)}{\partial x^2}\]
dabei bezeichnet $\kappa$ den Wärmeleitungskoefizienten.
\section{PDEs erster Ordnung}
\Comment{Diesen Abschnitt ausbauen. Wir sollten uns damit beschäftigen, denn man
kann auf die Maxwellschen Gleichungen und die Lösung mit der FDTD-Methode in
der Simulation zurückkommen.}

Wir werden uns nicht mit der Lösung von PDEs erster Ordnung beschäftigen,
da beliebige Systeme von PDEs erster Ordnung, welche komplexität oder
nichtlinearität sie auch aufweisen mögen, systematisch auf ein System
gekoppelter ODEs erster Ordnung zurückgeführt werden können.
Dies bedeutet: Gegeben eine beliebige Gleichung vom Typ $F(x,y;u,u_x,u_y)=0$
(Beachte, dass wir in der Liste die unabhängigen und abhängigen Variablen
durch ein Semikolon trennen), wobei $u_x$ ($u_y$) die partielle Ableitung der
gesuchten Funktion $u(x,y)$ nach der unabhängigen Variablen $x$ ($y$)
bezeichne. Diese Gleichung können wir auf ein System von ODEs transformieren.

Dies bedeutet nicht, dass das Resultierende System von ODEs analytisch lösbar
ist. In jedem Falle werden wir hier den Vorteil haben, dass der Formalismus zur
Lösung eines Systems von ODEs angewandt werden kann, entweder die Analytik
oder die Numerik.
\section{PDEs zweiter Ordnung}
Beispiele von PDEs zweiter Ordnung:\\
Wellengleichung
\begin{equation}
	\frac{\partial^2 u}{\partial x^2}-\frac{\partial^2 u}{\partial t^2}=0
\end{equation}
Diffusionsgleichung
\begin{equation}
	\frac{\partial^2 u}{\partial x^2}-\frac{\partial u}{\partial t}=0
\end{equation}
Laplacegleichung
\begin{equation}
	\frac{\partial^2 u}{\partial x^2}+\frac{\partial^2 u}{\partial y^2}=0
\end{equation}

Allgemeine Form linearer PDEs zweiter Ordnung.
\begin{equation}
	a(x,y) \frac{\partial^2 u}{\partial x^2}+
	b(x,y)\frac{\partial^2 u}{\partial x\partial y}+
	c(x,y)\frac{\partial^2 u}{\partial y^2}=F(x,y;u,u_x,u_y)
\end{equation}
Nimm an, dass $F=0$ und $a$, $b$, $c$ konstant seien.
\begin{equation}
        a\frac{\partial^2 u}{\partial x^2}+b\frac{\partial^2 u}{\partial x\partial y}+
        c\frac{\partial^2 u}{\partial y^2}=0
	\label{eqn2ndoconst}
\end{equation}
Mache den Ansatz $u(x,y)=f(mx+y)$ und setze in (\ref{eqn2ndoconst}) ein. Dies führt zu
\begin{eqnarray}
	\frac{\partial^2 u}{\partial x^2}&=&m^2f''\label{eqn2oxx}\\
	\frac{\partial^2 u}{\partial x\partial y}&=&mf''\label{eqn2oxy}\\
	\frac{\partial^2 u}{\partial y^2}&=&f'' \label{eqn2oyy}
\end{eqnarray}
Einsetzen von (\ref{eqn2oxx}-\ref{eqn2oyy}) in (\ref{eqn2ndoconst}) ergibt
\begin{equation}
	(am^2+bm+c)f''=0\label{eqndiscr}
\end{equation}
(\ref{eqndiscr}) hat zwei Faktoren, die Null sein können damit eine Lösung gefunden werden kann.
Diese zwei Fälle sind:
\begin{enumerate}
	\item $f''=0$ was $f(mx+y)=f_0+mx+y$ ergibt. Das ist nicht die allgemeinste Lösung.
	\item $am^2+bm+c=0$ ergibt zwei Lösungen für $m$
	\begin{equation}
		m_{1/2}=\frac{-b\pm\sqrt{b^2-4ac}}{2a}
		\label{eqnsol4m}
	\end{equation}
\end{enumerate}
Für (\ref{eqnsol4m}) haben wir drei Fälle zu unterscheiden:
\begin{description}
	\item[Der Fall $b^2-4ac>0$:] ergibt die so genannte hyperbolische PDE.\\ Es
		gibt nach (\ref{eqnsol4m}) zwei reelle Lösungen für $m$ in
		diesem Fall. Daher können wir die Lösung für
		$u(x,y)=F(m_1x+y)+G(m_2x+y)=F(\xi)+G(\eta)$ bestimmen. Mit
		$\xi=m_1x+y$ und $\eta=m_2x+y$ bekommen wir
		\begin{eqnarray}
			a\frac{\partial^2 u}{\partial x^2}&=&a\frac{\partial}{\partial x}\left(
			\frac{\partial u}{\partial\xi}\frac{\partial\xi}{\partial x}+
			\frac{\partial u}{\partial\eta}\frac{\partial\eta}{\partial x}\right)
			=a\frac{\partial}{\partial x}\left(
                        \frac{\partial u}{\partial\xi}m_1+\frac{\partial u}{\partial\eta}m_2\right)=\nonumber\\
			&=&a\left(\frac{\partial^2 u}{\partial\xi^2}m_1^2+
			2\frac{\partial^2 u}{\partial\xi\partial\eta}m_1m_2+
			\frac{\partial^2 u}{\partial\eta^2}m_2^2\right)\nonumber\\
			a\frac{\partial^2 u}{\partial x^2}&=&a\left(m_1\frac{\partial}{\partial\xi}+
			m_2\frac{\partial}{\partial\eta}\right)^2u\label{eqnad2udx2}
		\end{eqnarray}
	    	Für das gemischte Glied zweiter Ordnung erhalten wir
		\begin{eqnarray}
			b\frac{\partial^2 u}{\partial x\partial y}&=&b\left[
			\left(\frac{\partial^2 u}{\partial\xi^2}+
			      \frac{\partial^2 u}{\partial\xi\partial\eta}\right) m_1+
			\left(\frac{\partial^2 u}{\partial\eta^2}+
                              \frac{\partial^2 u}{\partial\xi\partial\eta}\right) m_2\right]\nonumber\\
			&=&-\frac{b^2}{a}\frac{\partial^2 u}{\partial\xi\partial\eta}+
			   bm_1\frac{\partial^2 u}{\partial\xi^2}+
			   bm_2\frac{\partial^2 u}{\partial\eta^2}\label{eqnd2udxdy}
		\end{eqnarray}
		Die zweite Ableitung nach $y$ lautet in den neuen Variablen $\xi$ und $\eta$
		\begin{equation}
			c\frac{\partial^2 u}{\partial y^2}=c\left[\frac{\partial}{\partial y}
			\left(\frac{\partial u}{\partial\xi}+\frac{\partial u}{\partial\eta}\right)\right]
			=c\left(\frac{\partial}{\partial\xi}+\frac{\partial}{\partial\eta}\right)^2u
			\label{eqnd2udy2}
		\end{equation}
		Eingesetzt in (\ref{eqn2ndoconst}) ergibt
		\begin{eqnarray}
			\lefteqn{
			\underbrace{\left(am_1^2+bm_1+c\right)}_{=0}
			\frac{\partial^2 u}{\partial\xi^2}+
			\underbrace{\left(am_2^2+bm_2+c\right)}_{=0}
			\frac{\partial^2 u}{\partial\eta^2}+
			}\nonumber\\[1ex]
			&&\left(2m_1m_2+bm_1+bm_2+2c\right)\frac{\partial^2 u}{\partial\xi\partial\eta}=0
			\label{eqn2ndoxieta}
		\end{eqnarray}
		Die ersten zwei Klammern in (\ref{eqn2ndoxieta}) verschwinden
		nach (\ref{eqndiscr}) und daher bleibt die Gleichung
		\begin{equation}
			\frac{\partial^2 u}{\partial\xi\partial\eta}=0
			\label{eqnd2udxideta0}
		\end{equation}
		Diese Gleichung ist äquivalent zu (\ref{eqn2ndoconst}) in den neuen Variablen
		$\xi$ und $\eta$. Integration von (\ref{eqnd2udxideta0}) nach $\xi$ und $\eta$ ergibt
		die Lösung $u(\xi,\eta)=F(\xi)+G(\eta)$, wobei $F$ und $G$
		beliebige Funktionen sind, die durch die Randbedingungen
		bestimmt sind und nur der Bedingung unterliegen, dass sie zwei
		mal differenzierbar sein müssen.
	\item[Der Fall $b^2-4ac=0$ mit $b\ne 0$ und $a\ne 0$:] führt zu einer parabolischen PDE.\\
		Es gibt nur eine Lösung für (\ref{eqndiscr}) und wir erhalten
		$m=-b/(2a)$. Wir unterziehen (\ref{eqn2ndoconst}) der folgenden
		Variablentransformation $\xi=m x + y$ and $\eta=y$ und erhalten
		\begin{equation}
			\left(am^2+bm+c\right)\frac{\partial^2 u}{\partial\xi^2}+
			(bm+2c)\frac{\partial^2 u}{\partial\xi\partial\eta}+
			c\frac{\partial^2 u}{\partial\eta^2}=0
			\label{eqntranfparab}
		\end{equation}
		Für $c\ne 0$ bekommen wir
		\begin{equation}
			\frac{\partial^2 u(\xi,\eta)}{\partial\eta^2}=0
			\label{eqnparab}
		\end{equation}
		Durch Integration bezüglich der Variablen $\eta$ erhalten
		wir die allgemeine Lösung von (\ref{eqnparab}) gegeben durch
		$u(\xi,\eta)=F(\xi)+\eta G(\xi)$ und welche in den ursprünglichen
		Variablen lautet:
		\begin{equation}
			u(x,y)=F(mx+y)+yG(mx+y)
			\label{eqnsolgenparab}
		\end{equation}
		Die eindimensionale Diffusionsgleichung ist ein Beispiel für
		eine parabolische PDE. Sie ist gegeben durch
		$a=b=0$. Beachte: die Diffusionsgleichung ist nicht in unserer
		Liste der Spezialfälle enthalten, da sie auch noch eine erste
		Ableitung enthält.
	\item[Der Fall $b^2-4ac<0$:] führt zu einer elliptischen PDE.\\
		Hier haben wir $m_2=m_1^*$ vorliegen. Die allgemeine Lösung
		ist damit gegeben als
		\begin{equation}
			u(x,y)=F(m_1x+y)+G(m_2x+y)=F(\xi)+G(\xi^*)
			\label{eqnsolell}
		\end{equation}
		Mit $\xi=v_1+iv_2$ erhalten wir die Gleichung
		\begin{equation}
			\frac{\partial^2 u}{\partial v_1^2}+\frac{\partial^2 u}{\partial v_2^2}=0
			\label{eqnelliptic}
		\end{equation}
		Eine Beispiel für eine elliptische Gleichung im
		Zweidimensionalen wie (\ref{eqnelliptic}) ist die
		Laplacegleichung.
\end{description}
Diese drei Typen linearer PDEs 2.\ Ordnung lassen sich für manche
Problemstellungen auch analytisch lösen. Wir geben im Folgenden ein
Beispiel hierzu.

\begin{example}{Wellengleichung}
Wir lösen die eindimensionale Wellengleichung.
\begin{equation}
	\frac{\partial^2 u}{\partial x^2}-\frac{1}{c^2}\frac{\partial^2 u}{\partial t^2}=0
	\label{eqn1Dwaveeqn}
\end{equation}
durch separation der Variablen. Dafür machen wir den Ansatz $u(x,t)=X(x)T(t)$, was zu
\begin{equation}
	\frac{1}{X}\frac{\partial^2 X}{\partial x^2}=\frac{1}{c^2}\frac{1}{T}\frac{\partial^2 T}{\partial t^2}
	\label{eqnseparate}
\end{equation}
führt.  In (\ref{eqnseparate}) hängt die linke Seite nur von der Variablen $x$ ab, während
die rechte Seite nur von $t$ abhängt. Für beliebige $x$ und $t$ kann diese
Gleichung nur erfüllt werden, wenn beide Seiten gleich einer Konstanten sind
und wir erhalten somit
\begin{equation}
        \frac{1}{X}\frac{\partial^2 X}{\partial x^2}=-k^2=\frac{1}{c^2}\frac{1}{T}\frac{\partial^2 T}{\partial t^2}
\end{equation}
Dies ergibt die folgenden zwei Gleichungen
\[\frac{\partial^2 X}{\partial x^2}+k^2X=0\]
mit der Lösung $X(x)=e^{\pm ikx}$ und
\[\frac{\partial^2 T}{\partial t^2}+\omega^2T=0\]
mit der Lösung $T(t)=e^{\pm i\omega t}$, wobei wir $\omega^2=c^2k^2$ gesetzt
haben.  Dieses Beispiel braucht zur Ergänzung Anfangsbedingungen, damit wir
eine Lösung finden können.
\end{example}
\subsection{Die Fundamentallösung oder Green'sche Funktion}
Die Idee hinter der Fundamentallösung stammt aus der linearen Antwortheorie

\begin{example}{Fundamentallösung der Diffusionsgleichung.}
Wir wenden uns im zweiten Beispiel der Untersuchung der eindimensionalen
Diffusionsgleichung zu. Wir wollen die Herleitung ihrer Fundamentallösung
etwas genauer betrachten und uns einige Anwendung anschauen.

Gegeben die Diffusionsgleichung in der Form
\begin{equation} 
	\frac{\partial u}{\partial t}-D\frac{\partial^2 u}{\partial x^2}=h(x,t)
	\label{eqndiffusion}
\end{equation}
wobei $D$ die Diffusionskonstante bezeichne und die Inhomogenität $h(x,t)$ eine bekannte Funktion sei. Man denke sich diese als raumzeitliche Quelle für $u(x,t)$.

Zunächst wollen wir $h(x,t)=0$ annehmen und die Anfangsbedingungen
$u(x,0)=f(x)$ festlegen. Nun wenden wir eine Fouriertransformation bezüglich
der Varaiablen $x$ an, der Raum sei unbegrenzt.  Bezüglich der Zeit $t$
wenden wir eine Laplacetransformation an, da wir ein Anfangswertproblem
vorliegen haben. Wir erhalten
\begin{equation}
	sU(k,s)-F(k)+k^2U(k,s)=0
	\label{eqntransfdiff}
\end{equation}
In transformierten Variablen erhalten wir die Lösung 
\[U(k,s)=\frac{F(k)}{s+k^2}\]
Die Funktion $G(k,s)=(s+k^2)^{-1}$ ist die Antwortfunktion des Systems im
Wellenzahlraum und im Frequenzbereich.
Die inverse Laplacetransformation ergibt
\[U(k,t)=e^{-k^2t}F(k)\]
Wir nennen $G(k,t)=e^{-k^2t}$ die Antwortfunktion im Raum der Wellenzahlen und
im Zeitbereich.  Um die Antwortfunktion zu bekommen, haben wir $F(k)=1$
gewählt, was äquivalent dazu ist, dass wir $f(x)=\delta(x)$ setzen. Der
letzte Schritt verlangt eine inverse Fouriertransformation und führt zu
\begin{equation}
	u(x,t)=\frac{1}{2\pi}\int_{-\infty}^\infty e^{-k^2t}e^{ikx}dk
	\label{eqninvfourierdiff}
\end{equation}
Quadratische Ergänzung im Exponenten führt zu
\begin{equation}
	-k^2t+ikx=-\left(k\sqrt{t}+\frac{ix}{2\sqrt{t}}\right)^2-\frac{x^2}{4t}
	\label{eqncomplsquare}
\end{equation}
Daher erhalten wir
\[
u(x,t)=\frac{e^{-\frac{x^2}{4t}}}{2\pi}\int_{-\infty}^\infty e^{-\left(k\sqrt{t}+\frac{ix}{2\sqrt{t}}\right)^2}dk=\frac{e^{-\frac{x^2}{4t}}}{\sqrt{4\pi t}}
\]
Die Antwortfunktion oder Greensche Funktion oder Fundamentallösung der Diffusionsgleichung ist daher
\begin{equation}
	g(x,t)=\frac{1}{\sqrt{4\pi t}}e^{-\frac{x^2}{4t}}
	\label{eqngreendiff}
\end{equation}
\end{example}

\begin{example}{Fundamentallösung der Wellengleichung}
Wir üben diese Vorgehensweise auch mit der Wellengleichung
\begin{equation}
	\frac{\partial^2 u}{\partial t^2}-\frac{\partial^2 u}{\partial x^2}=q(x,t)
	\label{eqnwave}
\end{equation}
Die benötigten Anfangsbedingungen lauten $u(x,0)=f(x)$ und $\left.\frac{\partial
u(x,t)}{\partial t}\right|_{t=0}=g(x)$. Die Fourier- und Laplacetransformierte
ergeben die Lösung im Frequenzbereich und dem Raum der Wellenzahlvektoren 
\begin{equation}
	U(k,s)=\frac{Q(k,s)}{s^2+k^2}+\frac{G(k)+sF(k)}{s^2+k^2}
	\label{eqnuofkands}
\end{equation}
Wir wählen $g(x)=f(x)=0$ und $q(x,t)=\delta(x-x')\delta(t-t')$. Die
Transformierte der Deltafunktionen lautet
\[ Q(k,s)=e^{ikx'-st'}\]
{\bf Achtung:} was bedeutet diese Wahl der Anfangsbedingungen und der Inhomogenität?

Nun müssen wir die inverse Fouriertransformation und die inverse Laplacetransformation anwenden, um die 
Greensche Funktion $G(x,t;x',t')$ zu bekommen und wir erhalten
\begin{equation}
	G(x,t;x',t')=\frac{1}{2}\left(H\left(t-t'-(x-x')\right)+H\left(t-t'+(x-x')\right)\right)=G(x-x',t-t')
	\label{eq:GFwave}
\end{equation}
\end{example}

\begin{example}{Fundamentallösung der Poissongleichung}
Verifiziere durch Anwendung der Fouriertransformation auf die Poissongleichung
\begin{equation}
	\nabla^2 V(\mbf{x})=-\rho(\mbf{x})
	\label{eq:Poisson}
\end{equation}
dass deren Fundamentallösung gegeben ist durch 
\begin{equation}
  G_0(\mbf{x},\mbf{x}')=\frac{1}{4\pi}\frac{1}{|\mbf{x}-\mbf{x}'|}
	\label{eq:FSPoisson}
\end{equation}
\end{example}
